\documentclass[12pt]{article}


\usepackage{arxiv}

\usepackage[utf8]{inputenc} % allow utf-8 input
\usepackage[T1]{fontenc}    % use 8-bit T1 fonts
\usepackage{hyperref}       % hyperlinks
\usepackage{url}            % simple URL typesetting
\usepackage{booktabs}       % professional-quality tables
\usepackage{amsfonts}       % blackboard math symbols
\usepackage{nicefrac}       % compact symbols for 1/2, etc.
\usepackage{microtype}      % microtypography
\usepackage{lipsum}
\usepackage{graphicx}

\title{Analysis of FAFSA completion rates and income in Maryland}


\author{
  Robert P Rose \\
  Department of Computer Science\\
  University of Maryland, Baltimore County\\
  Baltimore, MD 21250 \\
  \texttt{robrose2@umbc.edu} \\
  %% examples of more authors
   \And
  Ujjwal Rehani \\
  Department of Computer Science\\
  University of Maryland, Baltimore County\\
  Baltimore, MD 21250 \\
  \texttt{urehani1@umbc.edu} \\
}

\begin{document}
\maketitle

\begin{abstract}
In this paper we begin by processing and analyzing Free Application for Federal
Student Aid (FAFSA) completion and submission rates in Maryland. We then compared them to 
general statistics about schools taken from the National Center for Education 
Statistics database and with Maryland median income information on a 
county-by-county level. We predicted median income for the observation years using 
simple linear regression and attempt prediction of various FAFSA datapoints using 
varied machine and statistical learning methods. \\
\end{abstract}

\section{Introduction}
For our data science project, we chose to explore data outside the Open Baltimore
data library, as we found that it did not have the most interesting datasets. It
seemed that most of the analysis we could do with Open Baltimore was just crime
statistics analysis, which is a saturated field and full of misleading and 
loaded analysis.\cite{weatherburn2011} \\

Our first proposal was to use hospital location data and traffic accident records
to analyze if proximity to a trauma center increases someone's chance of survival.
Unfortunately it seemed that the information on Data.gov surrounding traffic
accidents was taken at the scene of the accident, and didn't provide long term
outcomes. As a result, any correlation we could find in the data would likely be
simply coincidence or the result of some interference variable. \\

We first chose data on FAFSA completion and submission rates. For our second 
dataset, we selected a dataset containing Maryland median income by county data.
This data contained the median income for all twenty four counties in Maryland 
from 2007 through 2016. We chose to limit our analysis to Maryland in order to 
not have a overbearing amount of data from the entire country. Finally, we also 
extracted information from a dataset containing information about every school 
in Maryland in order gain a better sense about the student body population that 
attended each school. \\


\section{Exploration of Data}
There were numerous predictor variables that we picked and thought were significant 
enough to use for our analysis. These predictor variables included whether a 
school was a magnet school, charter school, Title I school, Title I Wide school, 
and the locale of each school. Charter schools receive government funding but 
operate independently of the established state school system in which it is located. 
Magnet schools are public schools with specialized courses or curricula. Title I
schools have a student base that are lower-income and are provided with Title I 
funding in order to help those who are behind or at risk of falling behind, aiming 
to bridge the gap between low-income students and other students. Locale describes 
whether a schools is small, medium, large sized and what kind of area it is located 
in i.e. city, suburbs, etc. Count graphs for each of these variables were able to 
portray the spread of these categories. We found that the overwhelming amount of 
Maryland schools are not charter or magnet schools. There are a considerable amount 
of Title I and Title I wide schools in Maryland. The majority of schools tended to 
be classified as suburban and large.

\begin{figure}[!htb]
  \centering
  \includegraphics[width=150mm,scale=1.5]{title_1_wide_graph.png}
  \caption{Number of Title 1 School Wide. Cross represents missing data in the source.}
  \label{fig:title_1_wide}
\end{figure}

\section{Examples of citations, figures, tables, references}
\label{sec:others}
\lipsum[8] \cite{kour2014real,kour2014fast} and see \cite{hadash2018estimate}.

The documentation for \verb+natbib+ may be found at
\begin{center}
  \url{http://mirrors.ctan.org/macros/latex/contrib/natbib/natnotes.pdf}
\end{center}
Of note is the command \verb+\citet+, which produces citations
appropriate for use in inline text.  For example,
\begin{verbatim}
   \citet{hasselmo} investigated\dots
\end{verbatim}
produces
\begin{quote}
  Hasselmo, et al.\ (1995) investigated\dots
\end{quote}

\begin{center}
  \url{https://www.ctan.org/pkg/booktabs}
\end{center}


\subsection{Figures}
\lipsum[10] 
See Figure \ref{fig:title_1_wide}. Here is how you add footnotes. \footnote{Sample of the first footnote.}
\lipsum[11] 

\begin{figure}[!htb]
  \centering
  \includegraphics[width=150mm,scale=1.5]{Capture.PNG}
  \caption{Sample figure caption.}
  \label{fig:fig1}
\end{figure}

\subsection{Tables}
\lipsum[12]
See awesome Table~\ref{tab:table}.

\begin{table}
 \caption{Sample table title}
  \centering
  \begin{tabular}{lll}
    \toprule
    \multicolumn{2}{c}{Part}                   \\
    \cmidrule(r){1-2}
    Name     & Description     & Size ($\mu$m) \\
    \midrule
    Dendrite & Input terminal  & $\sim$100     \\
    Axon     & Output terminal & $\sim$10      \\
    Soma     & Cell body       & up to $10^6$  \\
    \bottomrule
  \end{tabular}
  \label{tab:table}
\end{table}

\subsection{Lists}
\begin{itemize}
\item Lorem ipsum dolor sit amet
\item consectetur adipiscing elit. 
\item Aliquam dignissim blandit est, in dictum tortor gravida eget. In ac rutrum magna.
\end{itemize}


\bibliographystyle{unsrt}  
\bibliography{references}  %%% Remove comment to use the external .bib file (using bibtex).
%%% and comment out the ``thebibliography'' section.


%%% Comment out this section when you \bibliography{references} is enabled.

\end{document}
