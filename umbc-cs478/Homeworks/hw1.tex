\documentclass[12pt]{article}
\setlength{\oddsidemargin}{0in}
\setlength{\evensidemargin}{0in}
\setlength{\textwidth}{6.5in}
\setlength{\parindent}{0in}
\setlength{\parskip}{\baselineskip}

\usepackage{amsmath,amsfonts,amssymb}
\usepackage{booktabs,multirow}

\title{CMSC 478 - HW1}

\begin{document}

CMSC 478 Fall 2018\hfill Homework \#1: Chapter 2.4\\
Robert Rose

\hrulefill

\begin{enumerate}
\item For each of parts (a) through (d), indicate whether we would generally
      expect the performance of a flexible statistical learning method to better
      better or worse than an inflexible method. Justify your answer.
  \begin{enumerate}
  \item The sample size $n$ is extremely large, and the number of predictors
        $p$ is small.\\
  \vspace{-2.5em}
  \paragraph{Better/Worse} Justification

  \item The number of predictors $p$ is extremely large, and the number of
        observations $n$ is small.\\
  \vspace{-2.5em}
  \paragraph{Better/Worse} Justificiation

  \item The relationship between the predictors and response is highly non-linear.\\
  \vspace{-2.5em}
  \paragraph{Better/Worse} Justification

  \item The variance of the error terms, i.e. $\sigma^2 = Var(\epsilon)$, is extremely high.\\
  \vspace{-2.5em}
  \paragraph{Better/Worse} Justification
  \end{enumerate}

\item Explain whether each scenario is a classification or regression problem,
      and indicate whether we are most interested in inference or prediction.
      Finally, provide $n$ and $p$.
  \begin{enumerate}
    \item We collect a set of data on the top 500 firms in the US. For each firm
          we record profit, number of employees, industry and the CEO salary. We
          are interested in understanding which factors affect CEO salary.\\
    \vspace{-2.5em}
    \paragraph{Regression} We're trying to model a number (CEO salary) rather than
          assign a single label. $n = 500, p = 4$

    \item We are considering launching a new product and wish to know whether it
          will be a \textit{success} or a \textit{failure}. We collect data on 20
          similar products that were previously launched. For each product we have
          recorded whether it was a success or failure, price charged for the product,
          marketing budget, competition price, and ten other variables.\\
    \vspace{-2.5em}
    \paragraph{Classification/Regression} Answer

    \item We are interested in predicting the \% change in the USD/Euro exchange rate
          in relation to the weekly changes in the world stock markets. Hence we collect
          weekly data for all of 2012. For each week we record the \% change in the
          USD/Euro, the \% change in the US market, the \% change in the British market,
          and the \% change in the German market.\\
    \vspace{-2.5em}
    \paragraph{Classification/Regression} Answer
  \end{enumerate}
\item We now revisit the bias-variance decomposition.
  \begin{enumerate}
    \item Provide a sketch of typical (squared) bias, variance, training error,
          test error, and Bayes (or irreducible) error curves, on a single plot,
          as we go from less flexible statistical learning methods towards more
          flexible approaches. The $x$-axis should represent the amount of
          flexibility in the method, and the y-axis should represent the values
          for each curve. There should be five curves. Make sure to label each one.
    \item Explain why each of the five curves has the shape displayed in part (a).
  \end{enumerate}
\item You will now think of some real-life applications for statistical learning.
  \begin{enumerate}
    \item Describe three real-life applications in which \textit{classification}
          might be useful. Describe the response, as well as the predictors. Is
          the goal of each application inference or prediction? Explain your answer.
    \item Describe three real-life applications in which \textit{regression} might
          be useful. Describe the response, as well as the predictors. Is the goal
          of each application inference or prediction? Explain your answer.
    \item Describe three real-life applications in which cluster analysis might
          be useful.
  \end{enumerate}
\item What are the advantages and disadvantages of a very flexible (versus a less
      flexible) approach for regression or classification? Under what circumstances
      might a more flexible approach be preferred to a less flexible approach?
      When might a less flexible approach be preferred?

\item Describe the differences between a parametric and a non-parametric statistical
      learning approach. What are the advantages of a parametric approach to regression
      or classification (as opposed to a nonparametric approach)? What are its
      disadvantages?

\item The table below provides a training data set containing six observations,
      three predictors, and one qualitative response variable. Suppose we wish
      to use this data set to make a prediction for $Y$ when $X_1 = X_2 = X_3 = 0$
      using $K$-nearest neighbors.

      \vspace{-1em}

      \begin{table}[h!]
        \begin{center}
          \begin{tabular}{l|cccl}
            Obs. & $X_1$ & $X_2$ & $X_3$ & $Y$ \\
            \hline
            1 & 0   & 3 & 0 & Red\\
            2 & 2   & 0 & 0 & Red\\
            3 & 0   & 1 & 3 & Red\\
            4 & 0   & 1 & 2 & Green\\
            5 & -1  & 0 & 1 & Green\\
            6 & 1   & 1 & 1 & Red
          \end{tabular}
        \end{center}
      \end{table}

      \vspace{-3em}

  \begin{enumerate}
    \item Compute the Euclidean distance between each observation and the test
          point, $X_1 = X_2 = X_3 = 0$.
    \item What is our prediction with $K = 1$? Why?
    \item What is our prediction with $K = 3$? Why?
    \item If the Bayes decision boundary in this problem is highly nonlinear,
          then would we expect the \textit{best} value for $K$ to be large or
          small? Why?
  \end{enumerate}

\end{enumerate}
\end{document}