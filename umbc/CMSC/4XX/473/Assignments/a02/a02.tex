\documentclass[12pt]{article}
\setlength{\oddsidemargin}{0in}
\setlength{\evensidemargin}{0in}
\setlength{\textwidth}{6.5in}
\setlength{\parindent}{0in}
\setlength{\parskip}{\baselineskip}

\usepackage{amsmath,amsfonts,amssymb}
\usepackage{booktabs,multirow}
\usepackage{graphicx}
\graphicspath{ {./} }

\title{CMSC 473 - A02}

\begin{document}

CMSC 473 Fall 2019\hfill Assignment \#2\\
Robert Rose

\hrulefill

\begin{enumerate}
\item Let $X$ and $Y$ be random variables where there's at least one pair of values $x$ and $y$
      such that $P(X = x, Y = y) = 0$. Show, with justification, whether $X$ and $Y$ are independent.
    \paragraph{Answer} No. They are not inependent because so long as X and Y are continuous,
                there is no way that $P(X = x, Y = y) = 0$ without them being dependent on
                one another. This is because the joint probability cannot be 0 unless one
                of them is 0.
\item Let $X_i$ be the random variable representing the $i$th roll of a six-sided die
      and let $x_i$ be the rolled (observed) value of the $i$th roll. Assume that we
      roll the die $N$ times and that each $X_i$ is an i.i.d. sample.
  \begin{enumerate}
  \item $Z = \frac{N}{6}$ because the dice has six sides, and we are rolling it $N$ times.
  \item Let's say that before we rolled our die, we first flipped a fair coin (2 outcomes, where
        probability of heads equals probability of tails). Call this random variable $Z_i$. 
        If $Z_i =$ H (heads), then we roll a fairly weighted six-sided die. If $Z_i =$ T (tails)
        then we roll a six-sided die where the probabilities of rolling an even number is
        double.
        \begin{enumerate}
            \item Write the conditional distributions $p(X|Z)$
                \begin{center}
                \begin{tabular}{ |c|c|c|c|c|c|c| }
                    \hline
                    $p(X|Z)$ & $X = 1$ & $X = 2$ & $X = 3$ & $X = 4$ & $X = 5$ & $X = 6$ \\
                    \hline
                    $Z = H$ & $\frac{1}{6}$ & $\frac{1}{6}$ & $\frac{1}{6}$ & $\frac{1}{6}$ & $\frac{1}{6}$ & $\frac{1}{6}$ \\
                    \hline
                    $Z = T$ & $\frac{1}{9}$ & $\frac{2}{9}$ & $\frac{1}{9}$ & $\frac{2}{9}$ & $\frac{1}{9}$ & $\frac{2}{9}$ \\
                    \hline
                \end{tabular}
                \end{center}
            \item Write the joint distributions $p(X|Z)$
                \begin{center}
                \begin{tabular}{ |c|c|c|c|c|c|c| }
                    \hline
                    $p(X,Z)$ & $X = 1$ & $X = 2$ & $X = 3$ & $X = 4$ & $X = 5$ & $X = 6$ \\
                    \hline
                    $Z = H$ & $\frac{1}{12}$ & $\frac{1}{12}$ & $\frac{1}{12}$ & $\frac{1}{12}$ & $\frac{1}{12}$ & $\frac{1}{12}$ \\
                    \hline
                    $Z = T$ & $\frac{1}{18}$ & $\frac{2}{18}$ & $\frac{1}{18}$ & $\frac{2}{18}$ & $\frac{1}{18}$ & $\frac{2}{18}$ \\
                    \hline
                \end{tabular}
                \end{center}
            \item Yes, $p(X, Z) = p(Z, X)$
            \item \begin{math}
                p(X_1 = 3) = \frac{5}{36} \\
                p(X_2 = 1) = \frac{5}{36} \\
                p(X_3 = 6) = \frac{7}{36} \\
                p(X_4 = 5) = \frac{5}{36} \\
                \frac{5}{36} \times \frac{5}{36} \times \frac{7}{36} \times \frac{5}{36} = \frac{875}{36^4}\\
            \end{math}
            \item $X$ and $Z$ are not independent because the probability vector of X is 
                  determined by the outcome of Z.
        \end{enumerate}
    \end{enumerate}

\item Read and review Church and Hanks (1989).
\paragraph{Review} The basis of Church and Hanks is the common theme of word co-occurence
        in linguistics. Although the relationships between words had long been conjectured,
        Church and Hanks proposed a more robust measure of word associations called the 
        association ratio. Since earlier studies and methods involving word co-occurence 
        were subjective and small scale (i.e. Palermo and Jenkins (1964)), association
        ratio is a measure that can be scaled up to an entire language. The basis of the
        assocaition ratio is their mutual information, that is, the probability of observing
        two words together (the join probability) with the probability of observing the two
        independently (chance). Church and Hanks go into detail in how they calculate the
        ratio and modifications they made to the ratio in order to have certain values be
        useful. Additionally they note that the order between two words matters greatly.

\item Zipfian languages
\paragraph{English and French} I used Python, numpy and seaborn to plot the following graphics,
        showing a linear regression line and a 95\% confidence interval. The confidence interval
        is so small it is hard to see in the PDF graphics. I did not get an $R^2$ value because
        I did the linear regression directly onto the graph using the seaborn library. 
        I would say it's unclear as to whether zipfian holds because so much of the data is
        clustered in one section of the graph. I'm not sure I did all the calculations correctly
        as a result.
        \begin{figure}[h]
            \caption{English}
            \centering
            \includegraphics[width=0.7\textwidth]{english.png}
        \end{figure}
        \begin{figure}[h]
            \caption{French}
            \centering
            \includegraphics[width=0.7\textwidth]{french.png}
        \end{figure}
\end{enumerate}
\end{document}