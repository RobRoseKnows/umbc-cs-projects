\documentclass[12pt]{article}
\setlength{\oddsidemargin}{0in}
\setlength{\evensidemargin}{0in}
\setlength{\textwidth}{6.5in}
\setlength{\parindent}{0in}
\setlength{\parskip}{\baselineskip}

\usepackage{amsmath,amsfonts,amssymb}
\usepackage{booktabs,multirow}
\usepackage{graphicx}

\title{CMSC 471 - Intro To ML HW}

\begin{document}

CMSC 471 Spring 2018\hfill Homework Intro To ML\\
Robert Rose

\hrulefill

The following table gives a data set for deciding whether to play or cancel a ball game, depending on the weather conditions:\\

\begin{center}
\begin{tabular}{l|l|l|l|l}
      \textbf{Outlook} & \textbf{Temp (F)} & \textbf{Humidity (\%)} & \textbf{Windy?} & \textbf{Class}\\
      \hline
      sunny & 75 & 70 & true & Play\\
      sunny & 80 & 90 & true & Don't Play\\
      sunny & 85 & 85 & false & Don't Play\\
        sunny & 72 & 95 & false & Don't Play\\
      sunny & 69 & 70 & false & Play\\
      overcast & 72 & 90 & true & Play\\
      overcast & 83 & 78 & false & Play\\
      overcast & 64 & 65 & true & Play\\
      overcast & 81 & 75 & false & Play\\
      rain & 71 & 80 & true & Don't Play\\
      rain & 65 & 70 & true & Don't Play\\
      rain & 75 & 80 & false & Play\\
      rain & 68 & 80 & false & Play\\
      rain & 70 & 96 & false & Play
\end{tabular}
\end{center}

\begin{enumerate}
\item \textbf{Information Gain} See next page for part (a). Had to separate it to give room for math.
\newpage
\begin{enumerate}
\item At the root node for a decision tree in this domain, what would the information gain associated with a split on the Outlook attribute?\\
\begin{equation*}
lg(x) = log_2(x)
\end{equation*}
\begin{equation*}
E = \sum_i - p_i lg(p_i)
\end{equation*}
Which using two attributes where $Y = \text{Class}$ and $X = \text{Attribute}$ is:\\
\begin{equation*}
E(Y, X) = \sum_{c \in X} P(c) E(c)
\end{equation*}
The formula for finding the information gain on a split on the Outlook is as follows:
\begin{equation*}
IG(Play, Outlook) = E(Play) - E(Play, Outlook)\\
\end{equation*}
So lets find all the parts and then plug it all in!
\begin{align*}
E(Play) &= - P(Play) \times lg(P(Play)) - P(\neg Play) \times lg(P(\neg Play))\\
&= - ((\frac{9}{14}) \times lg(\frac{9}{14})) - ((\frac{5}{14}) \times lg(\frac{5}{14}))\\
&= -.6428 \times lg(.6428) - .3571 \times lg(.3571)\\
&= .94\\
E(Play, Outlook) &= P(Sunny)E(Play, Sunny) + P(Rain)E(Play, Rain)\\
& + P(Overcast)E(Play, Overcast) \\
&= (\frac{5}{14}) \times .971 + (\frac{4}{14}) \times 0 + (\frac{5}{14}) \times .971\\
&= .3467 + 0 + .3467 = .6935\\
E(Play, Sunny) &= (\frac{3}{5})lg(\frac{3}{5}) + (\frac{2}{5})lg(\frac{2}{5})
= .6 \times lg(.6) + .4 \times lg(.4)
= .971\\
E(Play, Overcast) &= (\frac{4}{4})lg(\frac{4}{4}) + (\frac{0}{4})lg(\frac{0}{4})
= 1 \times lg(1)
= 0\\
E(Play, Rain) &= (\frac{3}{5})lg(\frac{3}{5}) + (\frac{2}{5})lg(\frac{2}{5})
= .6 \times lg(.6) + .4 \times lg(.4)
= .971\\
IG(Play, Outlook) &= .94 - .6935 = \boxed{.247}
\end{align*}

\newpage
\item What would it be for a split at the root on the Humidity attribute? (Use a threshold of 75 for humidity (i.e., assume a binary split: $humidity \leq 75$ or $humidity > 75$.)
Very similar to what we had last time, but this time with humidity instead. $E(Play)$ is even still the same!\\
\begin{align*}
E(Play) &= .94\\
E(Play, Humidity) &= P(H \leq 75)E(Play, H \leq 75) + P(H > 75)E(Play, H > 75)\\
&= .3571 \times .7219 + .6428 \times .9911 = .2578 + .6371\\
&= .8949\\
P(H \leq 75) &= \frac{5}{14} = .3571\\
P(H > 75) &= \frac{9}{14} = .6428\\
E(Play, H \leq 75) &= - (\frac{4}{5})lg(\frac{4}{5}) - (\frac{1}{5})lg(\frac{1}{5})\\
&= - .8 \times lg(.8) - .2 \times lg(.2)
= .2575 + .4644 = .7219\\
E(Play, H > 75) &= - (\frac{5}{9})lg(\frac{5}{9}) - (\frac{4}{9})lg(\frac{4}{9})\\
&= - .5556 \times lg(.5556) - .4444 \times lg(.4444)
= .4711 + .5199 = .9911\\
IG(Play, Humidity) &= E(Play) - E(Play, Humidity)\\
&= .94 - .8949 = \boxed{.0451}
\end{align*}

\end{enumerate}
\newpage

\item \textbf{Decision Tree} Suppose you build a decision tree that splits on the Outlook attribute at the root node.\\
\begin{enumerate}
\item How many children nodes are there are at the first level of the decision tree?\\
\vspace{-2em}
\paragraph{} There will be three children nodes on the first level, one for each of the possible outlooks, $Sunny$, $Overcast$ and $Rain$.\\
\item Which branches require a further split in order to create leaf nodes with instances belonging to a single class?\\
\vspace{-2em}
\paragraph{} The $Sunny$ and $Rain$ branches will require further splits in order to create single class leaf nodes. We do know however that all on the $Overcast$ branch will be $Play$ class however.
\item For each of these branches, which attribute can you split on to complete the decision tree building process at the next level (i.e., so that at level 2, there are only leaf nodes)?\\
\vspace{-2em}
\paragraph{Sunny} By splitting on the humidity attribute next, we can divide the remaining items into single class leaf nodes. All cases where $Humidity \leq 75$ (as proposed in part 1B above) will be classified as $Play$ while all cases where $Humidity > 75$ will be classified as $Don't Play$.

\paragraph{Rain} In order to divide the $Rain$ children into single class leaf nodes, we should split on $Windy?$ next. If $Windy = true$ then it should be classified as $Don't Play$, otherwise if $Windy = false$, it should be classified as $Play$.

\newpage
\item \textbf{Draw the resulting decision tree, showing the decisions (class predictions) at the leaves.} Yeah I'm going to draw this in on the printed out sheet.\\

\end{enumerate}

\end{enumerate}
\end{document}