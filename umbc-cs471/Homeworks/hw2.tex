\documentclass[12pt]{article}
\setlength{\oddsidemargin}{0in}
\setlength{\evensidemargin}{0in}
\setlength{\textwidth}{6.5in}
\setlength{\parindent}{0in}
\setlength{\parskip}{\baselineskip}

\usepackage{amsmath,amsfonts,amssymb}
\usepackage{booktabs,multirow}

\title{CMSC 471 - HW2}

\begin{document}

CMSC 471 Spring 2018\hfill Homework \#2\\
Robert Rose (02/07)

\hrulefill

\begin{enumerate}
\item You have an algorithm that is trying to classify pictures as either pictures of cats or dogs. The underlying distribution of these pictures is 30\% dogs and 70\% cats.
  \begin{enumerate}
  \item How good would your algorithm's accuracy have to be to do better than randomly guessing a cat or a dog?
    \begin{align*}
P(A\mid B) = \frac{P(B \mid A) \, P(A)}{P(B)}
    \end{align*}
  \item How good would your algorithm's accuracy have to be to do better than guessing only cats?
    \begin{align*}
P(A\mid B) = \frac{P(B \mid A) \, P(A)}{P(B)}
    \end{align*}
  \item How good would your algorithm's accuracy have to be to do better than guessing cats 70\% of the time and dogs 30\% of the time?
    \begin{align*}
P(A\mid B) = \frac{P(B \mid A) \, P(A)}{P(B)}
    \end{align*}
  \end{enumerate}
\newpage

\item Which lottery would you play, assuming the utility of a dollar is given by:
    \begin{equation*}
        Utility(\text{Prize}) = \log_2(\text{Prize} + 1)
    \end{equation*}

  \vspace{-1em}

  \begin{table}[h!]
    \begin{center}
      \caption{Lotteries}
      \label{tab:table1}
      \begin{tabular}{l|c|l}
        \textbf{Lottery} & \textbf{Odds} & \textbf{Prize}\\
        \hline
        A\textsubscript{low}    & 0.2 & \$8\\
        A\textsubscript{high}   & 0.8 & \$128\\
        B\textsubscript{low}    & 0.2 & \$2\\
        B\textsubscript{high}  & 0.8 & \$256\\
      \end{tabular}
    \end{center}
  \end{table}

  \vspace{-1em}

  \begin{enumerate}
  \item Which lottery would you play?
    \begin{align*}
        Utility(\text{A}) &= P(\text{A\textsubscript{low}}) \times Utility(\text{A\textsubscript{low}}) + P(\text{A\textsubscript{high}}) \times Utility(\text{A\textsubscript{high}})\\
        &= 0.2 \times \log_2(\text{A\textsubscript{low\textsubscript{prize}}} + 1) + 0.8 \times \log_2(\text{A\textsubscript{high\textsubscript{prize}}} + 1)\\
        &= 0.2 \times \log_2(\$8 + 1) + 0.8 \times \log_2(\$128 + 1)\\
        &= 0.2 \times \log_2(\$9) + 0.8 \times \log_2(\$129)\\
        &= 0.2 \times 3.170 + 0.8 \times 7.011\\
        &= .634 + 5.609\\
        &= 6.243\\\\
        Utility(\text{B}) &= P(\text{B\textsubscript{low}}) \times Utility(\text{B\textsubscript{low}}) + P(\text{B\textsubscript{high}}) \times Utility(\text{B\textsubscript{high}})\\
        &= 0.2 \times \log_2(\text{B\textsubscript{low\textsubscript{prize}}} + 1) + 0.8 \times \log_2(\text{B\textsubscript{high\textsubscript{prize}}} + 1)\\
        &= 0.2 \times \log_2(\$2 + 1) + 0.8 \times \log_2(\$256 + 1)\\
        &= 0.2 \times \log_2(\$3) + 0.8 \times \log_2(\$257)\\
        &= 0.2 \times 1.585 + 0.8 \times 8.006\\
        &= .317 + 6.404\\
        &= 6.721
    \end{align*}

    I would play the second lottery (Lottery B), because its overall utility is higher.\\
  \newpage
  \item What is the expected value of each lottery?
    \begin{align*}
        EV(\text{A}) &= P(\text{A\textsubscript{low}}) \times \text{A\textsubscript{low\textsubscript{prize}}} + P(\text{A\textsubscript{high}}) \times \text{A\textsubscript{high\textsubscript{prize}}}\\
        &= 0.2 \times \$8 + 0.8 \times \$128\\
        &= \boxed{\$104.00}\\\\
        EV(\text{B}) &= P(\text{B\textsubscript{low}}) \times \text{B\textsubscript{low\textsubscript{prize}}} + P(\text{B\textsubscript{high}}) \times \text{B\textsubscript{high\textsubscript{prize}}}\\
        &= 0.2 \times \$2 + 0.8 \times \$256\\
        &= \boxed{\$205.20}
    \end{align*}
  \end{enumerate}
\newpage

\item A bubonic plague test gives a positive result with probability 98\% when the patient is indeed affected by bubonic plague, while it gives a negative result with 99\% probability when the patient is not affected by bubonic plague. A patient is drawn at random from a population in which 0.1\% of individuals are affected by bubonic plague.
  \begin{enumerate}
  \item If he is found positive, what is the probability that he is indeed affected by bubonic plague?

    \begin{align*}
        P(\text{Infected}\mid\text{+}) &= \frac{P(\text{+}\mid\text{Infected}) P(\text{Infected})}{P(+)} \\
        &= \frac{P(\text{+}\mid\text{Infected}) P(\text{Infected})}{P(\text{+}\mid\text{Infected}) P(\text{Infected}) + P(\text{+}\mid\neg\text{Infected}) P(\neg\text{Infected})} \\[8pt]
        &= \frac{0.98 \times 0.001}{0.98 \times 0.001 + (1 - P(\text{--}\mid\neg\text{Infected})) \times (1 - P(\text{Infected}))} \\[8pt]
        &= \frac{0.98 \times 0.001}{0.98 \times 0.001 + (1 - 0.99) \times (1 - 0.001)} \\[8pt]
        &= \frac{0.98 \times 0.001}{0.98 \times 0.001 + 0.01 \times 0.999} \\[8pt]
        &\approx 8.933\%
    \end{align*}

  \end{enumerate}
\newpage

\item Question 4
  \begin{enumerate}
  \item Part A

  SOLUTION

  \item Part B

  SOLUTION
  \end{enumerate}
\newpage

\item Question 5
  \begin{enumerate}
  \item Part A

  SOLUTION

  \item Part B

  SOLUTION
  \end{enumerate}
\newpage

\end{enumerate}
\end{document}