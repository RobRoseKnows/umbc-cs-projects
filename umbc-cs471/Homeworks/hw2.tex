\documentclass[12pt]{article}
\setlength{\oddsidemargin}{0in}
\setlength{\evensidemargin}{0in}
\setlength{\textwidth}{6.5in}
\setlength{\parindent}{0in}
\setlength{\parskip}{\baselineskip}

\usepackage{amsmath,amsfonts,amssymb}
\usepackage{booktabs,multirow}

\title{CMSC 471 - HW2}

\begin{document}

%\begin{align*}
% P(A\mid B) = \frac{P(B \mid A) \, P(A)}{P(B)}
%\end{align*}

CMSC 471 Spring 2018\hfill Homework \#2\\
Robert Rose (02/07)

\hrulefill

\begin{enumerate}
\item You have an algorithm that is trying to classify pictures as either pictures of cats or dogs. The underlying distribution of these pictures is 30\% dogs and 70\% cats. Assuming accuracy refers to the percentage of images categorized correctly.
  \begin{enumerate}
  \item How good would your algorithm's accuracy have to be to do better than randomly guessing a cat or a dog?
    \vspace{-1em}
    \begin{table}[h!]
      \begin{center}
          \caption{Confusion Matrix for Random Classifier}
          \label{tab:table2}
          \begin{tabular}{l|c|c}
            & \textbf{Predicted Dog} & \textbf{Predicted Cat} \\
            \hline
            \textbf{Dog} & 0.15 & 0.15 \\
            \textbf{Cat} & 0.35 & 0.35 \\
          \end{tabular}
      \end{center}
    \end{table}

    \vspace{-2em}

    \begin{align*}
      P(\text{True Dog}) &= P(\text{Dog}) \times P(\text{Predicted Dog})\\
        &= 0.30 \times 0.50\\
        &= 0.15\\
        P(\text{True Cat}) &= P(\text{Cat}) \times P(\text{Predicted Cat})\\
        &= 0.70 \times 0.50\\
        &= 0.35\\
      P(\text{False Dog}) &= P(\text{Cat}) \times P(\text{Predicted Dog})\\
        &= 0.70 \times 0.50\\
        &= 0.35\\
        P(\text{False Cat}) &= P(\text{Dog}) \times P(\text{Predicted Cat})\\
        &= 0.30 \times 0.50\\
        &= 0.15\\
        ACC &= \frac{\text{True Dog} + \text{True Cat}}{\text{True Dog} + \text{True Cat} + \text{False Dog} + \text{False Cat}}\\[8pt]
        &= \frac{0.15 + 0.35}{0.15 + 0.35 + 0.35 + 0.15}\\[8pt]
        &= \boxed{50\%}
  \end{align*}

  \newpage

  \item How good would your algorithm's accuracy have to be to do better than guessing only cats?
  \vspace{-1em}
    \begin{table}[h!]
      \begin{center}
          \caption{Confusion Matrix for All-Cat Classifier}
          \label{tab:table2}
          \begin{tabular}{l|c|c}
            & \textbf{Predicted Dog} & \textbf{Predicted Cat} \\
            \hline
            \textbf{Dog} & 0.00 & 0.30 \\
            \textbf{Cat} & 0.00 & 0.70 \\
          \end{tabular}
      \end{center}
    \end{table}

    \vspace{-3em}

    \begin{align*}
      P(\text{True Dog}) &= P(\text{Dog}) \times P(\text{Predicted Dog})\\
        &= 0.30 \times 0.00\\
        &= 0.00\\
        P(\text{True Cat}) &= P(\text{Cat}) \times P(\text{Predicted Cat})\\
        &= 0.70 \times 1.00\\
        &= 0.70\\
      P(\text{False Dog}) &= P(\text{Cat}) \times P(\text{Predicted Dog})\\
        &= 0.70 \times 0.00\\
        &= 0.00\\
        P(\text{False Cat}) &= P(\text{Dog}) \times P(\text{Predicted Cat})\\
        &= 0.30 \times 1.00\\
        &= 0.30\\
        ACC &= \frac{\text{True Dog} + \text{True Cat}}{\text{True Dog} + \text{True Cat} + \text{False Dog} + \text{False Cat}}\\[8pt]
        &= \frac{0.00 + 0.70}{0.00 + 0.70 + 0.00 + 0.30}\\[8pt]
        &= \boxed{70\%}
  \end{align*}

  \newpage

  \item How good would your algorithm's accuracy have to be to do better than guessing cats 70\% of the time and dogs 30\% of the time?
  \vspace{-1em}
    \begin{table}[h!]
      \begin{center}
          \caption{Confusion Matrix for Distribution Classifier}
          \label{tab:table2}
          \begin{tabular}{l|c|c}
            & \textbf{Predicted Dog} & \textbf{Predicted Cat} \\
            \hline
            \textbf{Dog} & 0.09 & 0.21 \\
            \textbf{Cat} & 0.49 & 0.21 \\
          \end{tabular}
      \end{center}
    \end{table}

    \vspace{-3em}

    \begin{align*}
      P(\text{True Dog}) &= P(\text{Dog}) \times P(\text{Predicted Dog})\\
        &= 0.30 \times 0.30\\
        &= 0.09\\
        P(\text{True Cat}) &= P(\text{Cat}) \times P(\text{Predicted Cat})\\
        &= 0.70 \times 0.70\\
        &= 0.49\\
      P(\text{False Dog}) &= P(\text{Cat}) \times P(\text{Predicted Dog})\\
        &= 0.70 \times 0.30\\
        &= 0.21\\
        P(\text{False Cat}) &= P(\text{Dog}) \times P(\text{Predicted Cat})\\
        &= 0.30 \times 0.70\\
        &= 0.21\\
        ACC &= \frac{\text{True Dog} + \text{True Cat}}{\text{True Dog} + \text{True Cat} + \text{False Dog} + \text{False Cat}}\\[8pt]
        &= \frac{0.09 + 0.49}{0.09 + 0.49 + 0.21 + 0.21}\\[8pt]
        &= \boxed{58\%}
  \end{align*}
  \end{enumerate}
\newpage

\item Which lottery would you play, assuming the utility of a dollar is given by:
  \begin{equation*}
    Utility(\text{Prize}) = \log_2(\text{Prize} + 1)
  \end{equation*}

  \vspace{-2em}

  \begin{table}[h!]
    \begin{center}
      \caption{Lotteries}
      \label{tab:table1}
      \begin{tabular}{l|c|l}
        \textbf{Lottery} & \textbf{Odds} & \textbf{Prize}\\
        \hline
        A\textsubscript{low}  & 0.2 & \$8\\
        A\textsubscript{high}   & 0.8 & \$128\\
        B\textsubscript{low}  & 0.2 & \$2\\
        B\textsubscript{high}  & 0.8 & \$256\\
      \end{tabular}
    \end{center}
  \end{table}

  \vspace{-2em}

  \begin{enumerate}
  \item Which lottery would you play?
    \begin{align*}
    Utility(\text{A}) &= P(\text{A\textsubscript{low}}) \times Utility(\text{A\textsubscript{low}}) + P(\text{A\textsubscript{high}}) \times Utility(\text{A\textsubscript{high}})\\
        &= 0.2 \times \log_2(\text{A\textsubscript{low\textsubscript{prize}}} + 1) + 0.8 \times \log_2(\text{A\textsubscript{high\textsubscript{prize}}} + 1)\\
        &= 0.2 \times \log_2(\$8 + 1) + 0.8 \times \log_2(\$128 + 1)\\
        &= 0.2 \times \log_2(\$9) + 0.8 \times \log_2(\$129)\\
        &= 0.2 \times 3.170 + 0.8 \times 7.011\\
        &= .634 + 5.609\\
        &= 6.243\\\\
        Utility(\text{B}) &= P(\text{B\textsubscript{low}}) \times Utility(\text{B\textsubscript{low}}) + P(\text{B\textsubscript{high}}) \times Utility(\text{B\textsubscript{high}})\\
        &= 0.2 \times \log_2(\text{B\textsubscript{low\textsubscript{prize}}} + 1) + 0.8 \times \log_2(\text{B\textsubscript{high\textsubscript{prize}}} + 1)\\
        &= 0.2 \times \log_2(\$2 + 1) + 0.8 \times \log_2(\$256 + 1)\\
        &= 0.2 \times \log_2(\$3) + 0.8 \times \log_2(\$257)\\
        &= 0.2 \times 1.585 + 0.8 \times 8.006\\
        &= .317 + 6.404\\
        &= 6.721
  \end{align*}

    I would play the second lottery (Lottery B), because its overall utility is higher.\\
  \newpage
  \item What is the expected value of each lottery?
    \begin{align*}
      EV(\text{A}) &= P(\text{A\textsubscript{low}}) \times \text{A\textsubscript{low\textsubscript{prize}}} + P(\text{A\textsubscript{high}}) \times \text{A\textsubscript{high\textsubscript{prize}}}\\
        &= 0.2 \times \$8 + 0.8 \times \$128\\
        &= \boxed{\$104.00}\\\\
        EV(\text{B}) &= P(\text{B\textsubscript{low}}) \times \text{B\textsubscript{low\textsubscript{prize}}} + P(\text{B\textsubscript{high}}) \times \text{B\textsubscript{high\textsubscript{prize}}}\\
        &= 0.2 \times \$2 + 0.8 \times \$256\\
        &= \boxed{\$205.20}
  \end{align*}
  \end{enumerate}
\newpage

\item A bubonic plague test gives a positive result with probability 98\% when the patient is indeed affected by bubonic plague, while it gives a negative result with 99\% probability when the patient is not affected by bubonic plague. A patient is drawn at random from a population in which 0.1\% of individuals are affected by bubonic plague.
  \begin{enumerate}
  \item If he is found positive, what is the probability that he is indeed affected by bubonic plague?

  \begin{align*}
    P(\text{Infected}\mid\text{+}) &= \frac{P(\text{+}\mid\text{Infected}) P(\text{Infected})}{P(\text{+})} \\
    &= \frac{P(\text{+}\mid\text{Infected}) P(\text{Infected})}{P(\text{+}\mid\text{Infected})  P(\text{Infected}) + P(\text{+}\mid\neg\text{Infected}) P(\neg\text{Infected})} \\[8pt]
    &= \frac{0.98 \times 0.001}{0.98 \times 0.001 + (1 - P(\text{--}\mid\neg\text{Infected})) \times (1 - P(\text{Infected}))} \\[8pt]
    &= \frac{0.98 \times 0.001}{0.98 \times 0.001 + (1 - 0.99) \times (1 - 0.001)} \\[8pt]
        &= \frac{0.98 \times 0.001}{0.98 \times 0.001 + 0.01 \times 0.999} \\[8pt]
    &\approx \boxed{8.933\%}
  \end{align*}

  \end{enumerate}
\newpage

\item A friend who works in a big city owns two cars, one small and one large. Three-quarters of the time he drives the small car to work, and one-quarter of the time he drives the large car. If he takes the small car, he usually has little trouble parking, and so is at work on time with probability 0.9. If he takes the large car, he is at work on time with probability 0.6.
  \begin{enumerate}
  \item Given that he was on time on a particular morning, what is the probability that he drove the small car?

  \begin{align*}
    P(\text{Small}\mid\text{On Time}) &= \frac{P(\text{On Time}\mid\text{Small}) P(\text{Small})}{P(\text{On Time})} \\
    &= \frac{P(\text{On Time}\mid\text{Small}) P(\text{Small})}{P(\text{On Time}\mid\text{Small})  P(\text{Small}) + P(\text{On Time}\mid\text{Large})  P(\text{Large})} \\[8pt]
    &= \frac{0.90 \times 0.75}{0.90 \times 0.75 + 0.60 \times 0.25} \\[8pt]
    &= \frac{0.675}{0.675 + 0.15} \\[8pt]
        &= \frac{0.675}{0.825} \\[8pt]
    &= \boxed{81.\overline{81}\%}
  \end{align*}

  \end{enumerate}
\newpage

\item The following is a confusion matrix showing how well an algorithm is doing on a test set of a thousand samples:
  \vspace{-2em}

  \begin{table}[h!]
    \begin{center}
      \caption{Confusion Matrix}
      \label{tab:table2}
      \begin{tabular}{l|c|c}
         & \textbf{Predicted Positive} & \textbf{Predicted Negative}\\
        \hline
        \textbf{Positive} & 300 & 100\\
        \textbf{Negative} & 200 & 400\\
      \end{tabular}
    \end{center}
  \end{table}

  \vspace{-2em}
  \begin{enumerate}
  \item If your company makes \$5 per true positive, loses \$2 per false positive and \$1 per false negative (true negatives are revenue neutral) what is the expected profit on using this algorithm exactly \underline{\textbf{one}} time?

  \vspace{-1em}

    \begin{align*}
        P(\text{TP}) &= 0.30\\
        P(\text{FP}) &= 0.20\\
        P(\text{TN}) &= 0.40\\
        P(\text{FN}) &= 0.10\\\\
        E(\text{Profit}) &= \text{R\textsubscript{TP}} \times P(\text{TP}) + \text{R\textsubscript{FP}} \times P(\text{FP}) + \text{R\textsubscript{TN}} \times P(\text{TN}) + \text{R\textsubscript{FN}} \times P(\text{FN})\\
        &= (\$5 \times 0.30) + (-\$2 \times 0.20) + (\$0 \times 0.40) + (-\$1 \times 0.10)\\
        &= \$1.50 - \$0.40 - \$0.10\\
        &= \boxed{\$1.00}
    \end{align*}

  \end{enumerate}
\newpage

\end{enumerate}
\end{document}