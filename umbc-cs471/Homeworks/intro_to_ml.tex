\documentclass[12pt]{article}
\setlength{\oddsidemargin}{0in}
\setlength{\evensidemargin}{0in}
\setlength{\textwidth}{6.5in}
\setlength{\parindent}{0in}
\setlength{\parskip}{\baselineskip}

\usepackage{amsmath,amsfonts,amssymb}
\usepackage{booktabs,multirow}
\usepackage{graphicx}

\title{CMSC 471 - Intro To ML HW}

\begin{document}

CMSC 471 Spring 2018\hfill Homework Intro To ML\\
Robert Rose

\hrulefill

The following table gives a data set for deciding whether to play or cancel a ball game, depending on the weather conditions:\\

\begin{center}
\begin{tabular}{l|l|l|l|l}
      \textbf{Outlook} & \textbf{Temp (F)} & \textbf{Humidity (\%)} & \textbf{Windy?} & \textbf{Class}\\
      \hline
      sunny & 75 & 70 & true & Play\\
      sunny & 80 & 90 & true & Play\\
      sunny & 85 & 85 & false & Don't Play\\
      sunny & 72 & 95 & false & Don't Play\\
      sunny & 69 & 70 & false & Play\\
      overcast & 72 & 90 & true & Play\\
      overcast & 83 & 78 & false & Play\\
      overcast & 64 & 65 & true & Play\\
      overcast & 81 & 75 & false & Play\\
      rain & 71 & 80 & true & Don't Play\\
      rain & 65 & 70 & true & Don't Play\\
      rain & 75 & 80 & false & Play\\
      rain & 68 & 80 & false & Play\\
      rain & 70 & 96 & false & Play
\end{tabular}
\end{center}

\vspace{-2em}

\begin{enumerate}
\item \textbf{Information Gain} \\


Earthquake is \textbf{not independent} of alarm because the probability found in the table is not the same as the probability of $P(\text{alarm} \cap \text{earthquake})$ calculated as if they were independent.

\newpage

\end{enumerate}
\end{document}