
% Default to the notebook output style

    


% Inherit from the specified cell style.




    
\documentclass[11pt]{article}

    
    
    \usepackage[T1]{fontenc}
    % Nicer default font (+ math font) than Computer Modern for most use cases
    \usepackage{mathpazo}

    % Basic figure setup, for now with no caption control since it's done
    % automatically by Pandoc (which extracts ![](path) syntax from Markdown).
    \usepackage{graphicx}
    % We will generate all images so they have a width \maxwidth. This means
    % that they will get their normal width if they fit onto the page, but
    % are scaled down if they would overflow the margins.
    \makeatletter
    \def\maxwidth{\ifdim\Gin@nat@width>\linewidth\linewidth
    \else\Gin@nat@width\fi}
    \makeatother
    \let\Oldincludegraphics\includegraphics
    % Set max figure width to be 80% of text width, for now hardcoded.
    \renewcommand{\includegraphics}[1]{\Oldincludegraphics[width=.8\maxwidth]{#1}}
    % Ensure that by default, figures have no caption (until we provide a
    % proper Figure object with a Caption API and a way to capture that
    % in the conversion process - todo).
    \usepackage{caption}
    \DeclareCaptionLabelFormat{nolabel}{}
    \captionsetup{labelformat=nolabel}

    \usepackage{adjustbox} % Used to constrain images to a maximum size 
    \usepackage{xcolor} % Allow colors to be defined
    \usepackage{enumerate} % Needed for markdown enumerations to work
    \usepackage{geometry} % Used to adjust the document margins
    \usepackage{amsmath} % Equations
    \usepackage{amssymb} % Equations
    \usepackage{textcomp} % defines textquotesingle
    % Hack from http://tex.stackexchange.com/a/47451/13684:
    \AtBeginDocument{%
        \def\PYZsq{\textquotesingle}% Upright quotes in Pygmentized code
    }
    \usepackage{upquote} % Upright quotes for verbatim code
    \usepackage{eurosym} % defines \euro
    \usepackage[mathletters]{ucs} % Extended unicode (utf-8) support
    \usepackage[utf8x]{inputenc} % Allow utf-8 characters in the tex document
    \usepackage{fancyvrb} % verbatim replacement that allows latex
    \usepackage{grffile} % extends the file name processing of package graphics 
                         % to support a larger range 
    % The hyperref package gives us a pdf with properly built
    % internal navigation ('pdf bookmarks' for the table of contents,
    % internal cross-reference links, web links for URLs, etc.)
    \usepackage{hyperref}
    \usepackage{longtable} % longtable support required by pandoc >1.10
    \usepackage{booktabs}  % table support for pandoc > 1.12.2
    \usepackage[inline]{enumitem} % IRkernel/repr support (it uses the enumerate* environment)
    \usepackage[normalem]{ulem} % ulem is needed to support strikethroughs (\sout)
                                % normalem makes italics be italics, not underlines
    

    
    
    % Colors for the hyperref package
    \definecolor{urlcolor}{rgb}{0,.145,.698}
    \definecolor{linkcolor}{rgb}{.71,0.21,0.01}
    \definecolor{citecolor}{rgb}{.12,.54,.11}

    % ANSI colors
    \definecolor{ansi-black}{HTML}{3E424D}
    \definecolor{ansi-black-intense}{HTML}{282C36}
    \definecolor{ansi-red}{HTML}{E75C58}
    \definecolor{ansi-red-intense}{HTML}{B22B31}
    \definecolor{ansi-green}{HTML}{00A250}
    \definecolor{ansi-green-intense}{HTML}{007427}
    \definecolor{ansi-yellow}{HTML}{DDB62B}
    \definecolor{ansi-yellow-intense}{HTML}{B27D12}
    \definecolor{ansi-blue}{HTML}{208FFB}
    \definecolor{ansi-blue-intense}{HTML}{0065CA}
    \definecolor{ansi-magenta}{HTML}{D160C4}
    \definecolor{ansi-magenta-intense}{HTML}{A03196}
    \definecolor{ansi-cyan}{HTML}{60C6C8}
    \definecolor{ansi-cyan-intense}{HTML}{258F8F}
    \definecolor{ansi-white}{HTML}{C5C1B4}
    \definecolor{ansi-white-intense}{HTML}{A1A6B2}

    % commands and environments needed by pandoc snippets
    % extracted from the output of `pandoc -s`
    \providecommand{\tightlist}{%
      \setlength{\itemsep}{0pt}\setlength{\parskip}{0pt}}
    \DefineVerbatimEnvironment{Highlighting}{Verbatim}{commandchars=\\\{\}}
    % Add ',fontsize=\small' for more characters per line
    \newenvironment{Shaded}{}{}
    \newcommand{\KeywordTok}[1]{\textcolor[rgb]{0.00,0.44,0.13}{\textbf{{#1}}}}
    \newcommand{\DataTypeTok}[1]{\textcolor[rgb]{0.56,0.13,0.00}{{#1}}}
    \newcommand{\DecValTok}[1]{\textcolor[rgb]{0.25,0.63,0.44}{{#1}}}
    \newcommand{\BaseNTok}[1]{\textcolor[rgb]{0.25,0.63,0.44}{{#1}}}
    \newcommand{\FloatTok}[1]{\textcolor[rgb]{0.25,0.63,0.44}{{#1}}}
    \newcommand{\CharTok}[1]{\textcolor[rgb]{0.25,0.44,0.63}{{#1}}}
    \newcommand{\StringTok}[1]{\textcolor[rgb]{0.25,0.44,0.63}{{#1}}}
    \newcommand{\CommentTok}[1]{\textcolor[rgb]{0.38,0.63,0.69}{\textit{{#1}}}}
    \newcommand{\OtherTok}[1]{\textcolor[rgb]{0.00,0.44,0.13}{{#1}}}
    \newcommand{\AlertTok}[1]{\textcolor[rgb]{1.00,0.00,0.00}{\textbf{{#1}}}}
    \newcommand{\FunctionTok}[1]{\textcolor[rgb]{0.02,0.16,0.49}{{#1}}}
    \newcommand{\RegionMarkerTok}[1]{{#1}}
    \newcommand{\ErrorTok}[1]{\textcolor[rgb]{1.00,0.00,0.00}{\textbf{{#1}}}}
    \newcommand{\NormalTok}[1]{{#1}}
    
    % Additional commands for more recent versions of Pandoc
    \newcommand{\ConstantTok}[1]{\textcolor[rgb]{0.53,0.00,0.00}{{#1}}}
    \newcommand{\SpecialCharTok}[1]{\textcolor[rgb]{0.25,0.44,0.63}{{#1}}}
    \newcommand{\VerbatimStringTok}[1]{\textcolor[rgb]{0.25,0.44,0.63}{{#1}}}
    \newcommand{\SpecialStringTok}[1]{\textcolor[rgb]{0.73,0.40,0.53}{{#1}}}
    \newcommand{\ImportTok}[1]{{#1}}
    \newcommand{\DocumentationTok}[1]{\textcolor[rgb]{0.73,0.13,0.13}{\textit{{#1}}}}
    \newcommand{\AnnotationTok}[1]{\textcolor[rgb]{0.38,0.63,0.69}{\textbf{\textit{{#1}}}}}
    \newcommand{\CommentVarTok}[1]{\textcolor[rgb]{0.38,0.63,0.69}{\textbf{\textit{{#1}}}}}
    \newcommand{\VariableTok}[1]{\textcolor[rgb]{0.10,0.09,0.49}{{#1}}}
    \newcommand{\ControlFlowTok}[1]{\textcolor[rgb]{0.00,0.44,0.13}{\textbf{{#1}}}}
    \newcommand{\OperatorTok}[1]{\textcolor[rgb]{0.40,0.40,0.40}{{#1}}}
    \newcommand{\BuiltInTok}[1]{{#1}}
    \newcommand{\ExtensionTok}[1]{{#1}}
    \newcommand{\PreprocessorTok}[1]{\textcolor[rgb]{0.74,0.48,0.00}{{#1}}}
    \newcommand{\AttributeTok}[1]{\textcolor[rgb]{0.49,0.56,0.16}{{#1}}}
    \newcommand{\InformationTok}[1]{\textcolor[rgb]{0.38,0.63,0.69}{\textbf{\textit{{#1}}}}}
    \newcommand{\WarningTok}[1]{\textcolor[rgb]{0.38,0.63,0.69}{\textbf{\textit{{#1}}}}}
    
    
    % Define a nice break command that doesn't care if a line doesn't already
    % exist.
    \def\br{\hspace*{\fill} \\* }
    % Math Jax compatability definitions
    \def\gt{>}
    \def\lt{<}
    % Document parameters
    \title{titanic\_notebook}
    
    
    

    % Pygments definitions
    
\makeatletter
\def\PY@reset{\let\PY@it=\relax \let\PY@bf=\relax%
    \let\PY@ul=\relax \let\PY@tc=\relax%
    \let\PY@bc=\relax \let\PY@ff=\relax}
\def\PY@tok#1{\csname PY@tok@#1\endcsname}
\def\PY@toks#1+{\ifx\relax#1\empty\else%
    \PY@tok{#1}\expandafter\PY@toks\fi}
\def\PY@do#1{\PY@bc{\PY@tc{\PY@ul{%
    \PY@it{\PY@bf{\PY@ff{#1}}}}}}}
\def\PY#1#2{\PY@reset\PY@toks#1+\relax+\PY@do{#2}}

\expandafter\def\csname PY@tok@w\endcsname{\def\PY@tc##1{\textcolor[rgb]{0.73,0.73,0.73}{##1}}}
\expandafter\def\csname PY@tok@c\endcsname{\let\PY@it=\textit\def\PY@tc##1{\textcolor[rgb]{0.25,0.50,0.50}{##1}}}
\expandafter\def\csname PY@tok@cp\endcsname{\def\PY@tc##1{\textcolor[rgb]{0.74,0.48,0.00}{##1}}}
\expandafter\def\csname PY@tok@k\endcsname{\let\PY@bf=\textbf\def\PY@tc##1{\textcolor[rgb]{0.00,0.50,0.00}{##1}}}
\expandafter\def\csname PY@tok@kp\endcsname{\def\PY@tc##1{\textcolor[rgb]{0.00,0.50,0.00}{##1}}}
\expandafter\def\csname PY@tok@kt\endcsname{\def\PY@tc##1{\textcolor[rgb]{0.69,0.00,0.25}{##1}}}
\expandafter\def\csname PY@tok@o\endcsname{\def\PY@tc##1{\textcolor[rgb]{0.40,0.40,0.40}{##1}}}
\expandafter\def\csname PY@tok@ow\endcsname{\let\PY@bf=\textbf\def\PY@tc##1{\textcolor[rgb]{0.67,0.13,1.00}{##1}}}
\expandafter\def\csname PY@tok@nb\endcsname{\def\PY@tc##1{\textcolor[rgb]{0.00,0.50,0.00}{##1}}}
\expandafter\def\csname PY@tok@nf\endcsname{\def\PY@tc##1{\textcolor[rgb]{0.00,0.00,1.00}{##1}}}
\expandafter\def\csname PY@tok@nc\endcsname{\let\PY@bf=\textbf\def\PY@tc##1{\textcolor[rgb]{0.00,0.00,1.00}{##1}}}
\expandafter\def\csname PY@tok@nn\endcsname{\let\PY@bf=\textbf\def\PY@tc##1{\textcolor[rgb]{0.00,0.00,1.00}{##1}}}
\expandafter\def\csname PY@tok@ne\endcsname{\let\PY@bf=\textbf\def\PY@tc##1{\textcolor[rgb]{0.82,0.25,0.23}{##1}}}
\expandafter\def\csname PY@tok@nv\endcsname{\def\PY@tc##1{\textcolor[rgb]{0.10,0.09,0.49}{##1}}}
\expandafter\def\csname PY@tok@no\endcsname{\def\PY@tc##1{\textcolor[rgb]{0.53,0.00,0.00}{##1}}}
\expandafter\def\csname PY@tok@nl\endcsname{\def\PY@tc##1{\textcolor[rgb]{0.63,0.63,0.00}{##1}}}
\expandafter\def\csname PY@tok@ni\endcsname{\let\PY@bf=\textbf\def\PY@tc##1{\textcolor[rgb]{0.60,0.60,0.60}{##1}}}
\expandafter\def\csname PY@tok@na\endcsname{\def\PY@tc##1{\textcolor[rgb]{0.49,0.56,0.16}{##1}}}
\expandafter\def\csname PY@tok@nt\endcsname{\let\PY@bf=\textbf\def\PY@tc##1{\textcolor[rgb]{0.00,0.50,0.00}{##1}}}
\expandafter\def\csname PY@tok@nd\endcsname{\def\PY@tc##1{\textcolor[rgb]{0.67,0.13,1.00}{##1}}}
\expandafter\def\csname PY@tok@s\endcsname{\def\PY@tc##1{\textcolor[rgb]{0.73,0.13,0.13}{##1}}}
\expandafter\def\csname PY@tok@sd\endcsname{\let\PY@it=\textit\def\PY@tc##1{\textcolor[rgb]{0.73,0.13,0.13}{##1}}}
\expandafter\def\csname PY@tok@si\endcsname{\let\PY@bf=\textbf\def\PY@tc##1{\textcolor[rgb]{0.73,0.40,0.53}{##1}}}
\expandafter\def\csname PY@tok@se\endcsname{\let\PY@bf=\textbf\def\PY@tc##1{\textcolor[rgb]{0.73,0.40,0.13}{##1}}}
\expandafter\def\csname PY@tok@sr\endcsname{\def\PY@tc##1{\textcolor[rgb]{0.73,0.40,0.53}{##1}}}
\expandafter\def\csname PY@tok@ss\endcsname{\def\PY@tc##1{\textcolor[rgb]{0.10,0.09,0.49}{##1}}}
\expandafter\def\csname PY@tok@sx\endcsname{\def\PY@tc##1{\textcolor[rgb]{0.00,0.50,0.00}{##1}}}
\expandafter\def\csname PY@tok@m\endcsname{\def\PY@tc##1{\textcolor[rgb]{0.40,0.40,0.40}{##1}}}
\expandafter\def\csname PY@tok@gh\endcsname{\let\PY@bf=\textbf\def\PY@tc##1{\textcolor[rgb]{0.00,0.00,0.50}{##1}}}
\expandafter\def\csname PY@tok@gu\endcsname{\let\PY@bf=\textbf\def\PY@tc##1{\textcolor[rgb]{0.50,0.00,0.50}{##1}}}
\expandafter\def\csname PY@tok@gd\endcsname{\def\PY@tc##1{\textcolor[rgb]{0.63,0.00,0.00}{##1}}}
\expandafter\def\csname PY@tok@gi\endcsname{\def\PY@tc##1{\textcolor[rgb]{0.00,0.63,0.00}{##1}}}
\expandafter\def\csname PY@tok@gr\endcsname{\def\PY@tc##1{\textcolor[rgb]{1.00,0.00,0.00}{##1}}}
\expandafter\def\csname PY@tok@ge\endcsname{\let\PY@it=\textit}
\expandafter\def\csname PY@tok@gs\endcsname{\let\PY@bf=\textbf}
\expandafter\def\csname PY@tok@gp\endcsname{\let\PY@bf=\textbf\def\PY@tc##1{\textcolor[rgb]{0.00,0.00,0.50}{##1}}}
\expandafter\def\csname PY@tok@go\endcsname{\def\PY@tc##1{\textcolor[rgb]{0.53,0.53,0.53}{##1}}}
\expandafter\def\csname PY@tok@gt\endcsname{\def\PY@tc##1{\textcolor[rgb]{0.00,0.27,0.87}{##1}}}
\expandafter\def\csname PY@tok@err\endcsname{\def\PY@bc##1{\setlength{\fboxsep}{0pt}\fcolorbox[rgb]{1.00,0.00,0.00}{1,1,1}{\strut ##1}}}
\expandafter\def\csname PY@tok@kc\endcsname{\let\PY@bf=\textbf\def\PY@tc##1{\textcolor[rgb]{0.00,0.50,0.00}{##1}}}
\expandafter\def\csname PY@tok@kd\endcsname{\let\PY@bf=\textbf\def\PY@tc##1{\textcolor[rgb]{0.00,0.50,0.00}{##1}}}
\expandafter\def\csname PY@tok@kn\endcsname{\let\PY@bf=\textbf\def\PY@tc##1{\textcolor[rgb]{0.00,0.50,0.00}{##1}}}
\expandafter\def\csname PY@tok@kr\endcsname{\let\PY@bf=\textbf\def\PY@tc##1{\textcolor[rgb]{0.00,0.50,0.00}{##1}}}
\expandafter\def\csname PY@tok@bp\endcsname{\def\PY@tc##1{\textcolor[rgb]{0.00,0.50,0.00}{##1}}}
\expandafter\def\csname PY@tok@fm\endcsname{\def\PY@tc##1{\textcolor[rgb]{0.00,0.00,1.00}{##1}}}
\expandafter\def\csname PY@tok@vc\endcsname{\def\PY@tc##1{\textcolor[rgb]{0.10,0.09,0.49}{##1}}}
\expandafter\def\csname PY@tok@vg\endcsname{\def\PY@tc##1{\textcolor[rgb]{0.10,0.09,0.49}{##1}}}
\expandafter\def\csname PY@tok@vi\endcsname{\def\PY@tc##1{\textcolor[rgb]{0.10,0.09,0.49}{##1}}}
\expandafter\def\csname PY@tok@vm\endcsname{\def\PY@tc##1{\textcolor[rgb]{0.10,0.09,0.49}{##1}}}
\expandafter\def\csname PY@tok@sa\endcsname{\def\PY@tc##1{\textcolor[rgb]{0.73,0.13,0.13}{##1}}}
\expandafter\def\csname PY@tok@sb\endcsname{\def\PY@tc##1{\textcolor[rgb]{0.73,0.13,0.13}{##1}}}
\expandafter\def\csname PY@tok@sc\endcsname{\def\PY@tc##1{\textcolor[rgb]{0.73,0.13,0.13}{##1}}}
\expandafter\def\csname PY@tok@dl\endcsname{\def\PY@tc##1{\textcolor[rgb]{0.73,0.13,0.13}{##1}}}
\expandafter\def\csname PY@tok@s2\endcsname{\def\PY@tc##1{\textcolor[rgb]{0.73,0.13,0.13}{##1}}}
\expandafter\def\csname PY@tok@sh\endcsname{\def\PY@tc##1{\textcolor[rgb]{0.73,0.13,0.13}{##1}}}
\expandafter\def\csname PY@tok@s1\endcsname{\def\PY@tc##1{\textcolor[rgb]{0.73,0.13,0.13}{##1}}}
\expandafter\def\csname PY@tok@mb\endcsname{\def\PY@tc##1{\textcolor[rgb]{0.40,0.40,0.40}{##1}}}
\expandafter\def\csname PY@tok@mf\endcsname{\def\PY@tc##1{\textcolor[rgb]{0.40,0.40,0.40}{##1}}}
\expandafter\def\csname PY@tok@mh\endcsname{\def\PY@tc##1{\textcolor[rgb]{0.40,0.40,0.40}{##1}}}
\expandafter\def\csname PY@tok@mi\endcsname{\def\PY@tc##1{\textcolor[rgb]{0.40,0.40,0.40}{##1}}}
\expandafter\def\csname PY@tok@il\endcsname{\def\PY@tc##1{\textcolor[rgb]{0.40,0.40,0.40}{##1}}}
\expandafter\def\csname PY@tok@mo\endcsname{\def\PY@tc##1{\textcolor[rgb]{0.40,0.40,0.40}{##1}}}
\expandafter\def\csname PY@tok@ch\endcsname{\let\PY@it=\textit\def\PY@tc##1{\textcolor[rgb]{0.25,0.50,0.50}{##1}}}
\expandafter\def\csname PY@tok@cm\endcsname{\let\PY@it=\textit\def\PY@tc##1{\textcolor[rgb]{0.25,0.50,0.50}{##1}}}
\expandafter\def\csname PY@tok@cpf\endcsname{\let\PY@it=\textit\def\PY@tc##1{\textcolor[rgb]{0.25,0.50,0.50}{##1}}}
\expandafter\def\csname PY@tok@c1\endcsname{\let\PY@it=\textit\def\PY@tc##1{\textcolor[rgb]{0.25,0.50,0.50}{##1}}}
\expandafter\def\csname PY@tok@cs\endcsname{\let\PY@it=\textit\def\PY@tc##1{\textcolor[rgb]{0.25,0.50,0.50}{##1}}}

\def\PYZbs{\char`\\}
\def\PYZus{\char`\_}
\def\PYZob{\char`\{}
\def\PYZcb{\char`\}}
\def\PYZca{\char`\^}
\def\PYZam{\char`\&}
\def\PYZlt{\char`\<}
\def\PYZgt{\char`\>}
\def\PYZsh{\char`\#}
\def\PYZpc{\char`\%}
\def\PYZdl{\char`\$}
\def\PYZhy{\char`\-}
\def\PYZsq{\char`\'}
\def\PYZdq{\char`\"}
\def\PYZti{\char`\~}
% for compatibility with earlier versions
\def\PYZat{@}
\def\PYZlb{[}
\def\PYZrb{]}
\makeatother


    % Exact colors from NB
    \definecolor{incolor}{rgb}{0.0, 0.0, 0.5}
    \definecolor{outcolor}{rgb}{0.545, 0.0, 0.0}



    
    % Prevent overflowing lines due to hard-to-break entities
    \sloppy 
    % Setup hyperref package
    \hypersetup{
      breaklinks=true,  % so long urls are correctly broken across lines
      colorlinks=true,
      urlcolor=urlcolor,
      linkcolor=linkcolor,
      citecolor=citecolor,
      }
    % Slightly bigger margins than the latex defaults
    
    \geometry{verbose,tmargin=1in,bmargin=1in,lmargin=1in,rmargin=1in}
    
    

    \begin{document}
    
    
    \maketitle
    
    

    
    \begin{Verbatim}[commandchars=\\\{\}]
{\color{incolor}In [{\color{incolor}1}]:} \PY{k+kn}{import} \PY{n+nn}{pandas} \PY{k}{as} \PY{n+nn}{pd}
        \PY{k+kn}{import} \PY{n+nn}{numpy} \PY{k}{as} \PY{n+nn}{np}
        \PY{k+kn}{import} \PY{n+nn}{matplotlib}\PY{n+nn}{.}\PY{n+nn}{pyplot} \PY{k}{as} \PY{n+nn}{plt}
        \PY{k+kn}{from} \PY{n+nn}{IPython}\PY{n+nn}{.}\PY{n+nn}{display} \PY{k}{import} \PY{n}{display}\PY{p}{,} \PY{n}{HTML}
\end{Verbatim}


    \section{UMBC CS 471 Project 3: Titanic Survivors with Random
Forests}\label{umbc-cs-471-project-3-titanic-survivors-with-random-forests}

Using a Jupyter notebook like a boss. Already imported pandas and numpy,
will import sklearn later. Also the display function for Jupyter
notebooks.

\subsection{Preprocessing}\label{preprocessing}

First thing I had to do was to actually load the data into a dataframe.
It's a good thing Pandas has a function just for this. Then just display
the head in order to make sure it read in everything.

    \begin{Verbatim}[commandchars=\\\{\}]
{\color{incolor}In [{\color{incolor}2}]:} \PY{c+c1}{\PYZsh{} Data file is in the same file as the rest.}
        \PY{n}{file\PYZus{}name} \PY{o}{=} \PY{l+s+s2}{\PYZdq{}}\PY{l+s+s2}{./train.csv}\PY{l+s+s2}{\PYZdq{}}
        \PY{n}{df} \PY{o}{=} \PY{n}{pd}\PY{o}{.}\PY{n}{read\PYZus{}csv}\PY{p}{(}\PY{n}{file\PYZus{}name}\PY{p}{)}
        
        \PY{c+c1}{\PYZsh{} Display the head.}
        \PY{n}{display}\PY{p}{(}\PY{n}{df}\PY{o}{.}\PY{n}{head}\PY{p}{(}\PY{p}{)}\PY{p}{)}
\end{Verbatim}


    
    \begin{verbatim}
   PassengerId  Survived  Pclass  \
0            1         0       3   
1            2         1       1   
2            3         1       3   
3            4         1       1   
4            5         0       3   

                                                Name     Sex   Age  SibSp  \
0                            Braund, Mr. Owen Harris    male  22.0      1   
1  Cumings, Mrs. John Bradley (Florence Briggs Th...  female  38.0      1   
2                             Heikkinen, Miss. Laina  female  26.0      0   
3       Futrelle, Mrs. Jacques Heath (Lily May Peel)  female  35.0      1   
4                           Allen, Mr. William Henry    male  35.0      0   

   Parch            Ticket     Fare Cabin Embarked  
0      0         A/5 21171   7.2500   NaN        S  
1      0          PC 17599  71.2833   C85        C  
2      0  STON/O2. 3101282   7.9250   NaN        S  
3      0            113803  53.1000  C123        S  
4      0            373450   8.0500   NaN        S  
    \end{verbatim}

    
    Nice so it looks like things loaded in correctly. I looked at an old
project I did with the same data set to get what each value meant:

\begin{itemize}
\tightlist
\item
  \textbf{Survived:} Outcome of survival (0 = No; 1 = Yes)
\item
  \textbf{Pclass:} Socio-economic class (1 = Upper class; 2 = Middle
  class; 3 = Lower class)
\item
  \textbf{Name:} Name of passenger
\item
  \textbf{Sex:} Sex of the passenger
\item
  \textbf{Age:} Age of the passenger (Some entries contain NaN)
\item
  \textbf{SibSp:} Number of siblings and spouses of the passenger aboard
\item
  \textbf{Parch:} Number of parents and children of the passenger aboard
\item
  \textbf{Ticket:} Ticket number of the passenger
\item
  \textbf{Fare:} Fare paid by the passenger
\item
  \textbf{Cabin:} Cabin number of the passenger (Some entries contain
  NaN)
\item
  \textbf{Embarked:} Port of embarkation of the passenger (C =
  Cherbourg; Q = Queenstown; S = Southampton)
\end{itemize}

Since the \textbf{Survived} column is a boolean, lets tell pandas to
treat it like one and take a look at the new data frame

    \begin{Verbatim}[commandchars=\\\{\}]
{\color{incolor}In [{\color{incolor}3}]:} \PY{n}{df} \PY{o}{=} \PY{n}{df}\PY{o}{.}\PY{n}{astype}\PY{p}{(}\PY{p}{\PYZob{}}\PY{l+s+s1}{\PYZsq{}}\PY{l+s+s1}{Survived}\PY{l+s+s1}{\PYZsq{}}\PY{p}{:} \PY{n+nb}{bool}\PY{p}{\PYZcb{}}\PY{p}{)}
        \PY{n}{display}\PY{p}{(}\PY{n}{df}\PY{o}{.}\PY{n}{head}\PY{p}{(}\PY{p}{)}\PY{p}{)}
\end{Verbatim}


    
    \begin{verbatim}
   PassengerId  Survived  Pclass  \
0            1     False       3   
1            2      True       1   
2            3      True       3   
3            4      True       1   
4            5     False       3   

                                                Name     Sex   Age  SibSp  \
0                            Braund, Mr. Owen Harris    male  22.0      1   
1  Cumings, Mrs. John Bradley (Florence Briggs Th...  female  38.0      1   
2                             Heikkinen, Miss. Laina  female  26.0      0   
3       Futrelle, Mrs. Jacques Heath (Lily May Peel)  female  35.0      1   
4                           Allen, Mr. William Henry    male  35.0      0   

   Parch            Ticket     Fare Cabin Embarked  
0      0         A/5 21171   7.2500   NaN        S  
1      0          PC 17599  71.2833   C85        C  
2      0  STON/O2. 3101282   7.9250   NaN        S  
3      0            113803  53.1000  C123        S  
4      0            373450   8.0500   NaN        S  
    \end{verbatim}

    
    Nice! That worked exactly how I wanted it to!

To get a better idea of what the data looks like, lets describe each
column's data. That should tell us about any NaN's and whatnot. There
are 891 datapoints, so any columns with \textless{} 891 data points have
missing data. I'm going to both \texttt{count()} the values in each
column and use \texttt{describe()} on all the ones not having
\texttt{NaN} as the \textbf{Age}, since from previous looks,
\textbf{Age} is the only continuous one with missing data.

    \begin{Verbatim}[commandchars=\\\{\}]
{\color{incolor}In [{\color{incolor}4}]:} \PY{n}{df}\PY{o}{.}\PY{n}{loc}\PY{p}{[}\PY{n}{df}\PY{p}{[}\PY{l+s+s2}{\PYZdq{}}\PY{l+s+s2}{Age}\PY{l+s+s2}{\PYZdq{}}\PY{p}{]}\PY{o}{.}\PY{n}{isnull}\PY{p}{(}\PY{p}{)} \PY{o}{==} \PY{k+kc}{False}\PY{p}{]}\PY{o}{.}\PY{n}{describe}\PY{p}{(}\PY{p}{)}
\end{Verbatim}


\begin{Verbatim}[commandchars=\\\{\}]
{\color{outcolor}Out[{\color{outcolor}4}]:}        PassengerId      Pclass         Age       SibSp       Parch        Fare
        count   714.000000  714.000000  714.000000  714.000000  714.000000  714.000000
        mean    448.582633    2.236695   29.699118    0.512605    0.431373   34.694514
        std     259.119524    0.838250   14.526497    0.929783    0.853289   52.918930
        min       1.000000    1.000000    0.420000    0.000000    0.000000    0.000000
        25\%     222.250000    1.000000   20.125000    0.000000    0.000000    8.050000
        50\%     445.000000    2.000000   28.000000    0.000000    0.000000   15.741700
        75\%     677.750000    3.000000   38.000000    1.000000    1.000000   33.375000
        max     891.000000    3.000000   80.000000    5.000000    6.000000  512.329200
\end{Verbatim}
            
    \begin{Verbatim}[commandchars=\\\{\}]
{\color{incolor}In [{\color{incolor}5}]:} \PY{n}{df}\PY{o}{.}\PY{n}{count}\PY{p}{(}\PY{p}{)}
\end{Verbatim}


\begin{Verbatim}[commandchars=\\\{\}]
{\color{outcolor}Out[{\color{outcolor}5}]:} PassengerId    891
        Survived       891
        Pclass         891
        Name           891
        Sex            891
        Age            714
        SibSp          891
        Parch          891
        Ticket         891
        Fare           891
        Cabin          204
        Embarked       889
        dtype: int64
\end{Verbatim}
            
    Nice, so the only ones with missing data are \textbf{Age},
\textbf{Cabin} and \textbf{Embarked}. That's pretty good. I do find it
kind of weird that the minimum age is a float, especially one that isn't
a divisor of 12. Kind of weird but whatever.

For convenience sake, I want to create a nice little function though
that will tell me how many are missing though, so I can use it again.
(Got this idea from
\href{https://www.analyticsvidhya.com/blog/2016/01/12-pandas-techniques-python-data-manipulation/}{here}).

Then I'm going to use it on each column (via apply) and print the
results, to get a baseline value. Future iterations can just call
\texttt{print\_missing} with the frame in order to check missing values
in each column.

    \begin{Verbatim}[commandchars=\\\{\}]
{\color{incolor}In [{\color{incolor}6}]:} \PY{k}{def} \PY{n+nf}{num\PYZus{}missing}\PY{p}{(}\PY{n}{x}\PY{p}{)} \PY{o}{\PYZhy{}}\PY{o}{\PYZgt{}} \PY{n+nb}{int}\PY{p}{:}
            \PY{k}{return} \PY{n+nb}{sum}\PY{p}{(}\PY{n}{x}\PY{o}{.}\PY{n}{isnull}\PY{p}{(}\PY{p}{)}\PY{p}{)}
        
        \PY{k}{def} \PY{n+nf}{print\PYZus{}missing}\PY{p}{(}\PY{n}{frame}\PY{p}{)}\PY{p}{:}
            \PY{n+nb}{print}\PY{p}{(}\PY{n}{frame}\PY{o}{.}\PY{n}{apply}\PY{p}{(}\PY{n}{num\PYZus{}missing}\PY{p}{,} \PY{n}{axis}\PY{o}{=}\PY{l+m+mi}{0}\PY{p}{)}\PY{p}{)}
            
        \PY{n}{print\PYZus{}missing}\PY{p}{(}\PY{n}{df}\PY{p}{)}
\end{Verbatim}


    \begin{Verbatim}[commandchars=\\\{\}]
PassengerId      0
Survived         0
Pclass           0
Name             0
Sex              0
Age            177
SibSp            0
Parch            0
Ticket           0
Fare             0
Cabin          687
Embarked         2
dtype: int64

    \end{Verbatim}

    Nice, that confirms what I already thought about the missing data based
on the earlier count.

\subsubsection{Embarked}\label{embarked}

But I am curious as to why only two embarked values are missing. I'm
going to use \texttt{loc{[}{]}} to see what's special about them.

    \begin{Verbatim}[commandchars=\\\{\}]
{\color{incolor}In [{\color{incolor}7}]:} \PY{n}{df}\PY{o}{.}\PY{n}{loc}\PY{p}{[}\PY{n}{df}\PY{p}{[}\PY{l+s+s2}{\PYZdq{}}\PY{l+s+s2}{Embarked}\PY{l+s+s2}{\PYZdq{}}\PY{p}{]}\PY{o}{.}\PY{n}{isnull}\PY{p}{(}\PY{p}{)}\PY{p}{]}
\end{Verbatim}


\begin{Verbatim}[commandchars=\\\{\}]
{\color{outcolor}Out[{\color{outcolor}7}]:}      PassengerId  Survived  Pclass                                       Name  \textbackslash{}
        61            62      True       1                        Icard, Miss. Amelie   
        829          830      True       1  Stone, Mrs. George Nelson (Martha Evelyn)   
        
                Sex   Age  SibSp  Parch  Ticket  Fare Cabin Embarked  
        61   female  38.0      0      0  113572  80.0   B28      NaN  
        829  female  62.0      0      0  113572  80.0   B28      NaN  
\end{Verbatim}
            
    Okay that's weird. They were both in the same cabin and ticket. If we
were making a really long decision tree, Embarked would almost certainly
be a split, but would probably overfit.

In order to prevent errors later on with \texttt{NaN}, I'm just going to
set the point of embarktation for those two as \texttt{U} for Unknown.

    \begin{Verbatim}[commandchars=\\\{\}]
{\color{incolor}In [{\color{incolor}8}]:} \PY{n}{df} \PY{o}{=} \PY{n}{df}\PY{o}{.}\PY{n}{fillna}\PY{p}{(}\PY{p}{\PYZob{}}\PY{l+s+s2}{\PYZdq{}}\PY{l+s+s2}{Embarked}\PY{l+s+s2}{\PYZdq{}}\PY{p}{:} \PY{l+s+s2}{\PYZdq{}}\PY{l+s+s2}{U}\PY{l+s+s2}{\PYZdq{}}\PY{p}{\PYZcb{}}\PY{p}{)}
        \PY{n}{print\PYZus{}missing}\PY{p}{(}\PY{n}{df}\PY{p}{)}
\end{Verbatim}


    \begin{Verbatim}[commandchars=\\\{\}]
PassengerId      0
Survived         0
Pclass           0
Name             0
Sex              0
Age            177
SibSp            0
Parch            0
Ticket           0
Fare             0
Cabin          687
Embarked         0
dtype: int64

    \end{Verbatim}

    Nice! No more missing values in \textbf{Embarked}! I believe it's safe
to insert dummy values for Embarked, because it's already a category as
is. \textbf{Age} might be harder since it's a continuous variable, but
first I want to look at \textbf{Cabin}, which are missing \emph{a lot}
but probably have a lot of juicy info. (In fact I don't think I will do
anything for age, as it's potentially valuable to know that someone has
no age on record.

\subsubsection{Cabin}\label{cabin}

The first few \textbf{Cabin} values all start with a letter, but I don't
want to take that for granted, others might have different values. I'm
going to do some weird trickery here in order to analyze this. First I
want to get a pandas Series containing only non-NaN cabin values.

    \begin{Verbatim}[commandchars=\\\{\}]
{\color{incolor}In [{\color{incolor}9}]:} \PY{n}{non\PYZus{}nan\PYZus{}cabins} \PY{o}{=} \PY{n}{df}\PY{o}{.}\PY{n}{loc}\PY{p}{[}\PY{n}{df}\PY{p}{[}\PY{l+s+s2}{\PYZdq{}}\PY{l+s+s2}{Cabin}\PY{l+s+s2}{\PYZdq{}}\PY{p}{]}\PY{o}{.}\PY{n}{isnull}\PY{p}{(}\PY{p}{)} \PY{o}{==} \PY{k+kc}{False}\PY{p}{]}\PY{p}{[}\PY{l+s+s2}{\PYZdq{}}\PY{l+s+s2}{Cabin}\PY{l+s+s2}{\PYZdq{}}\PY{p}{]}
        \PY{n}{non\PYZus{}nan\PYZus{}cabins}\PY{o}{.}\PY{n}{count}\PY{p}{(}\PY{p}{)}
\end{Verbatim}


\begin{Verbatim}[commandchars=\\\{\}]
{\color{outcolor}Out[{\color{outcolor}9}]:} 204
\end{Verbatim}
            
    Nice! That matches what we know should be there. Our earlier count of
the non-null values of \textbf{Cabin} returned 204, so that's what we
were hoping for! Now lets see what all that data looks like. Thankfully,
pandas provides a nice vectorized function that should let me run a
regex over each part of the series. I can then describe that data, or
even make a nice histogram!

There's a couple things I want to test from a brief look at the CSV in
Excel: * Most Cabins seem to be in the format with a letter and then one
or more numbers, so I'll start by counting the number of cabin numbers
matching that format in each row. * I'll check ones that are letters
only. * I'll check ones that are numbers only. * I also want to check
which ones have lower case letters (if any at all).

    \begin{Verbatim}[commandchars=\\\{\}]
{\color{incolor}In [{\color{incolor}10}]:} \PY{c+c1}{\PYZsh{} First lets import python\PYZsq{}s regex module.}
         \PY{k+kn}{import} \PY{n+nn}{re}
         
         \PY{c+c1}{\PYZsh{} Standard regex format (letter + numbers)}
         \PY{n}{std\PYZus{}format\PYZus{}regex} \PY{o}{=} \PY{n}{re}\PY{o}{.}\PY{n}{compile}\PY{p}{(}\PY{l+s+sa}{r}\PY{l+s+s1}{\PYZsq{}}\PY{l+s+s1}{([A\PYZhy{}Z]+)([0\PYZhy{}9]+)}\PY{l+s+s1}{\PYZsq{}}\PY{p}{)}
         \PY{n}{std\PYZus{}format\PYZus{}count} \PY{o}{=} \PY{n}{non\PYZus{}nan\PYZus{}cabins}\PY{o}{.}\PY{n}{str}\PY{o}{.}\PY{n}{count}\PY{p}{(}\PY{n}{std\PYZus{}format\PYZus{}regex}\PY{p}{)}
         \PY{n}{display}\PY{p}{(}\PY{n}{std\PYZus{}format\PYZus{}count}\PY{o}{.}\PY{n}{describe}\PY{p}{(}\PY{p}{)}\PY{p}{)}
         
         \PY{c+c1}{\PYZsh{} Letters but no numbers.}
         \PY{n}{no\PYZus{}num\PYZus{}regex} \PY{o}{=} \PY{n}{re}\PY{o}{.}\PY{n}{compile}\PY{p}{(}\PY{l+s+sa}{r}\PY{l+s+s1}{\PYZsq{}}\PY{l+s+s1}{\PYZbs{}}\PY{l+s+s1}{b([A\PYZhy{}Z]+)(?![A\PYZhy{}Z]*[0\PYZhy{}9]+)}\PY{l+s+s1}{\PYZsq{}}\PY{p}{)}
         \PY{n}{no\PYZus{}num\PYZus{}count} \PY{o}{=} \PY{n}{non\PYZus{}nan\PYZus{}cabins}\PY{o}{.}\PY{n}{str}\PY{o}{.}\PY{n}{count}\PY{p}{(}\PY{n}{no\PYZus{}num\PYZus{}regex}\PY{p}{)}
         \PY{n}{display}\PY{p}{(}\PY{n}{no\PYZus{}num\PYZus{}count}\PY{o}{.}\PY{n}{describe}\PY{p}{(}\PY{p}{)}\PY{p}{)}
                 
         \PY{c+c1}{\PYZsh{} Numbers but no letters}
         \PY{n}{no\PYZus{}let\PYZus{}regex} \PY{o}{=} \PY{n}{re}\PY{o}{.}\PY{n}{compile}\PY{p}{(}\PY{l+s+sa}{r}\PY{l+s+s1}{\PYZsq{}}\PY{l+s+s1}{\PYZbs{}}\PY{l+s+s1}{b([0\PYZhy{}9]+)(?![0\PYZhy{}9]*[A\PYZhy{}Z]+)}\PY{l+s+s1}{\PYZsq{}}\PY{p}{)}
         \PY{n}{no\PYZus{}let\PYZus{}count} \PY{o}{=} \PY{n}{non\PYZus{}nan\PYZus{}cabins}\PY{o}{.}\PY{n}{str}\PY{o}{.}\PY{n}{count}\PY{p}{(}\PY{n}{no\PYZus{}let\PYZus{}regex}\PY{p}{)}
         \PY{n}{display}\PY{p}{(}\PY{n}{no\PYZus{}let\PYZus{}count}\PY{o}{.}\PY{n}{describe}\PY{p}{(}\PY{p}{)}\PY{p}{)}
\end{Verbatim}


    
    \begin{verbatim}
count    204.000000
mean       1.127451
std        0.519028
min        0.000000
25%        1.000000
50%        1.000000
75%        1.000000
max        4.000000
Name: Cabin, dtype: float64
    \end{verbatim}

    
    
    \begin{verbatim}
count    204.000000
mean       0.039216
std        0.194585
min        0.000000
25%        0.000000
50%        0.000000
75%        0.000000
max        1.000000
Name: Cabin, dtype: float64
    \end{verbatim}

    
    
    \begin{verbatim}
count    204.0
mean       0.0
std        0.0
min        0.0
25%        0.0
50%        0.0
75%        0.0
max        0.0
Name: Cabin, dtype: float64
    \end{verbatim}

    
    The \texttt{describe()} function wasn't exactly as helpful as I would've
liked, so lets try a different thing using \texttt{count()} and
\texttt{loc()}. I'm not particularly interested in the means and what
not, I more so just want to make sure everything is accounted for. Which
means removing non-zeros. Though since the third regex looks like it was
unneccessary, we can omit that for future tests!

    \begin{Verbatim}[commandchars=\\\{\}]
{\color{incolor}In [{\color{incolor}11}]:} \PY{c+c1}{\PYZsh{} Lets get a chunk of the nonzero entries.}
         \PY{n}{std\PYZus{}non\PYZus{}zero} \PY{o}{=} \PY{n}{std\PYZus{}format\PYZus{}count}\PY{o}{.}\PY{n}{nonzero}\PY{p}{(}\PY{p}{)}
         \PY{n}{nn\PYZus{}non\PYZus{}zero} \PY{o}{=} \PY{n}{no\PYZus{}num\PYZus{}count}\PY{o}{.}\PY{n}{nonzero}\PY{p}{(}\PY{p}{)}
         
         \PY{c+c1}{\PYZsh{} Check to see how many got triggered on each}
         \PY{n+nb}{print}\PY{p}{(}\PY{n+nb}{len}\PY{p}{(}\PY{n}{std\PYZus{}non\PYZus{}zero}\PY{p}{[}\PY{l+m+mi}{0}\PY{p}{]}\PY{p}{)}\PY{p}{)}
         \PY{n+nb}{print}\PY{p}{(}\PY{n+nb}{len}\PY{p}{(}\PY{n}{nn\PYZus{}non\PYZus{}zero}\PY{p}{[}\PY{l+m+mi}{0}\PY{p}{]}\PY{p}{)}\PY{p}{)}
\end{Verbatim}


    \begin{Verbatim}[commandchars=\\\{\}]
200
8

    \end{Verbatim}

    That's unexpected... those shouldn't have totaled to more than 204.
Looks like the regex isn't perfect, or the data is werider than I
thought. Lets turn those into sets and take the union.

    \begin{Verbatim}[commandchars=\\\{\}]
{\color{incolor}In [{\color{incolor}12}]:} \PY{n}{snz\PYZus{}set} \PY{o}{=} \PY{n+nb}{set}\PY{p}{(}\PY{n}{std\PYZus{}non\PYZus{}zero}\PY{p}{[}\PY{l+m+mi}{0}\PY{p}{]}\PY{p}{)}
         \PY{n}{nnnz\PYZus{}set} \PY{o}{=} \PY{n+nb}{set}\PY{p}{(}\PY{n}{nn\PYZus{}non\PYZus{}zero}\PY{p}{[}\PY{l+m+mi}{0}\PY{p}{]}\PY{p}{)}
         
         \PY{n+nb}{print}\PY{p}{(}\PY{l+s+s2}{\PYZdq{}}\PY{l+s+s2}{Std non zero set len:}\PY{l+s+s2}{\PYZdq{}}\PY{p}{,} \PY{n+nb}{len}\PY{p}{(}\PY{n}{snz\PYZus{}set}\PY{p}{)}\PY{p}{)}
         \PY{n+nb}{print}\PY{p}{(}\PY{l+s+s2}{\PYZdq{}}\PY{l+s+s2}{No number non zero set len:}\PY{l+s+s2}{\PYZdq{}}\PY{p}{,} \PY{n+nb}{len}\PY{p}{(}\PY{n}{nnnz\PYZus{}set}\PY{p}{)}\PY{p}{)}
         
         \PY{n}{common} \PY{o}{=} \PY{n}{snz\PYZus{}set} \PY{o}{\PYZam{}} \PY{n}{nnnz\PYZus{}set}
         \PY{n+nb}{print}\PY{p}{(}\PY{n}{common}\PY{p}{)}
\end{Verbatim}


    \begin{Verbatim}[commandchars=\\\{\}]
Std non zero set len: 200
No number non zero set len: 8
\{25, 163, 156, 15\}

    \end{Verbatim}

    So the \texttt{nonzero} function would have returned the index of the
values, so we can just look at those rows of \texttt{non\_nan\_cabins}
indivually to see what's special about them.

    \begin{Verbatim}[commandchars=\\\{\}]
{\color{incolor}In [{\color{incolor}13}]:} \PY{n}{display}\PY{p}{(}\PY{n}{non\PYZus{}nan\PYZus{}cabins}\PY{o}{.}\PY{n}{iloc}\PY{p}{[}\PY{n+nb}{list}\PY{p}{(}\PY{n}{common}\PY{p}{)}\PY{p}{]}\PY{p}{)}
\end{Verbatim}


    
    \begin{verbatim}
128    F E69
715    F G73
699    F G63
75     F G73
Name: Cabin, dtype: object
    \end{verbatim}

    
    Oh. So it just had both letter only and standard format cabins. I'm not
sure if the \texttt{F} is part of the cabin description, but that may
not matter. I'm going to create a histogram of each of the counts, as
well as print out all the columns with lone letters, to help give me a
better idea of what to do with them.

    \begin{Verbatim}[commandchars=\\\{\}]
{\color{incolor}In [{\color{incolor}14}]:} \PY{n}{std\PYZus{}format\PYZus{}count}\PY{o}{.}\PY{n}{hist}\PY{p}{(}\PY{p}{)}
         \PY{n}{no\PYZus{}num\PYZus{}count}\PY{o}{.}\PY{n}{hist}\PY{p}{(}\PY{p}{)}
         
         \PY{n}{display}\PY{p}{(}\PY{n}{non\PYZus{}nan\PYZus{}cabins}\PY{o}{.}\PY{n}{iloc}\PY{p}{[}\PY{n+nb}{list}\PY{p}{(}\PY{n}{nnnz\PYZus{}set}\PY{p}{)}\PY{p}{]}\PY{p}{)}
\end{Verbatim}


    
    \begin{verbatim}
715    F G73
327        D
473        D
339        T
75     F G73
128    F E69
292        D
699    F G63
Name: Cabin, dtype: object
    \end{verbatim}

    
    \begin{center}
    \adjustimage{max size={0.9\linewidth}{0.9\paperheight}}{output_25_1.png}
    \end{center}
    { \hspace*{\fill} \\}
    
    Nice, that gives me a better idea of what I'm working with. From looking
at the sole letters, I'm not sure what to make of what those mean, but I
do think I know how I want to categorize the data. (I'll add a few more
columns to the table with categories of number of cabins, cabin letter
and single letter cabins. We can do more pandas stuff after that.

    \begin{Verbatim}[commandchars=\\\{\}]
{\color{incolor}In [{\color{incolor}15}]:} \PY{n}{df}\PY{p}{[}\PY{l+s+s2}{\PYZdq{}}\PY{l+s+s2}{num\PYZus{}std\PYZus{}cabins}\PY{l+s+s2}{\PYZdq{}}\PY{p}{]} \PY{o}{=} \PY{n}{std\PYZus{}format\PYZus{}count}
         \PY{n}{df}\PY{p}{[}\PY{l+s+s2}{\PYZdq{}}\PY{l+s+s2}{num\PYZus{}let\PYZus{}cabins}\PY{l+s+s2}{\PYZdq{}}\PY{p}{]} \PY{o}{=} \PY{n}{no\PYZus{}num\PYZus{}count}
         
         \PY{n}{df} \PY{o}{=} \PY{n}{df}\PY{o}{.}\PY{n}{fillna}\PY{p}{(}\PY{p}{\PYZob{}}\PY{l+s+s2}{\PYZdq{}}\PY{l+s+s2}{num\PYZus{}std\PYZus{}cabins}\PY{l+s+s2}{\PYZdq{}}\PY{p}{:} \PY{o}{\PYZhy{}}\PY{l+m+mi}{1}\PY{p}{,} \PY{l+s+s2}{\PYZdq{}}\PY{l+s+s2}{num\PYZus{}let\PYZus{}cabins}\PY{l+s+s2}{\PYZdq{}}\PY{p}{:} \PY{o}{\PYZhy{}}\PY{l+m+mi}{1}\PY{p}{\PYZcb{}}\PY{p}{)}
         
         \PY{n}{df}\PY{p}{[}\PY{l+s+s2}{\PYZdq{}}\PY{l+s+s2}{num\PYZus{}std\PYZus{}cabins}\PY{l+s+s2}{\PYZdq{}}\PY{p}{]} \PY{o}{=} \PY{n}{df}\PY{p}{[}\PY{l+s+s2}{\PYZdq{}}\PY{l+s+s2}{num\PYZus{}std\PYZus{}cabins}\PY{l+s+s2}{\PYZdq{}}\PY{p}{]}\PY{o}{.}\PY{n}{astype}\PY{p}{(}\PY{n+nb}{int}\PY{p}{)}
         \PY{n}{df}\PY{p}{[}\PY{l+s+s2}{\PYZdq{}}\PY{l+s+s2}{num\PYZus{}let\PYZus{}cabins}\PY{l+s+s2}{\PYZdq{}}\PY{p}{]} \PY{o}{=} \PY{n}{df}\PY{p}{[}\PY{l+s+s2}{\PYZdq{}}\PY{l+s+s2}{num\PYZus{}let\PYZus{}cabins}\PY{l+s+s2}{\PYZdq{}}\PY{p}{]}\PY{o}{.}\PY{n}{astype}\PY{p}{(}\PY{n+nb}{int}\PY{p}{)}
         
         \PY{n}{display}\PY{p}{(}\PY{n}{df}\PY{o}{.}\PY{n}{head}\PY{p}{(}\PY{p}{)}\PY{p}{)}
         
         \PY{n}{display}\PY{p}{(}\PY{n}{df}\PY{o}{.}\PY{n}{loc}\PY{p}{[}\PY{n}{df}\PY{p}{[}\PY{l+s+s1}{\PYZsq{}}\PY{l+s+s1}{num\PYZus{}std\PYZus{}cabins}\PY{l+s+s1}{\PYZsq{}}\PY{p}{]} \PY{o}{\PYZgt{}} \PY{l+m+mi}{1}\PY{p}{]}\PY{p}{)}
\end{Verbatim}


    
    \begin{verbatim}
   PassengerId  Survived  Pclass  \
0            1     False       3   
1            2      True       1   
2            3      True       3   
3            4      True       1   
4            5     False       3   

                                                Name     Sex   Age  SibSp  \
0                            Braund, Mr. Owen Harris    male  22.0      1   
1  Cumings, Mrs. John Bradley (Florence Briggs Th...  female  38.0      1   
2                             Heikkinen, Miss. Laina  female  26.0      0   
3       Futrelle, Mrs. Jacques Heath (Lily May Peel)  female  35.0      1   
4                           Allen, Mr. William Henry    male  35.0      0   

   Parch            Ticket     Fare Cabin Embarked  num_std_cabins  \
0      0         A/5 21171   7.2500   NaN        S              -1   
1      0          PC 17599  71.2833   C85        C               1   
2      0  STON/O2. 3101282   7.9250   NaN        S              -1   
3      0            113803  53.1000  C123        S               1   
4      0            373450   8.0500   NaN        S              -1   

   num_let_cabins  
0              -1  
1               0  
2              -1  
3               0  
4              -1  
    \end{verbatim}

    
    
    \begin{verbatim}
     PassengerId  Survived  Pclass  \
27            28     False       1   
88            89      True       1   
97            98      True       1   
118          119     False       1   
297          298     False       1   
299          300      True       1   
305          306      True       1   
311          312      True       1   
341          342      True       1   
390          391      True       1   
435          436      True       1   
438          439     False       1   
498          499     False       1   
679          680      True       1   
700          701      True       1   
742          743      True       1   
763          764      True       1   
789          790     False       1   
802          803      True       1   
872          873     False       1   

                                                  Name     Sex    Age  SibSp  \
27                      Fortune, Mr. Charles Alexander    male  19.00      3   
88                          Fortune, Miss. Mabel Helen  female  23.00      3   
97                     Greenfield, Mr. William Bertram    male  23.00      0   
118                           Baxter, Mr. Quigg Edmond    male  24.00      0   
297                       Allison, Miss. Helen Loraine  female   2.00      1   
299    Baxter, Mrs. James (Helene DeLaudeniere Chaput)  female  50.00      0   
305                     Allison, Master. Hudson Trevor    male   0.92      1   
311                         Ryerson, Miss. Emily Borie  female  18.00      2   
341                     Fortune, Miss. Alice Elizabeth  female  24.00      3   
390                         Carter, Mr. William Ernest    male  36.00      1   
435                          Carter, Miss. Lucile Polk  female  14.00      1   
438                                  Fortune, Mr. Mark    male  64.00      1   
498    Allison, Mrs. Hudson J C (Bessie Waldo Daniels)  female  25.00      1   
679                 Cardeza, Mr. Thomas Drake Martinez    male  36.00      0   
700  Astor, Mrs. John Jacob (Madeleine Talmadge Force)  female  18.00      1   
742              Ryerson, Miss. Susan Parker "Suzette"  female  21.00      2   
763          Carter, Mrs. William Ernest (Lucile Polk)  female  36.00      1   
789                           Guggenheim, Mr. Benjamin    male  46.00      0   
802                Carter, Master. William Thornton II    male  11.00      1   
872                           Carlsson, Mr. Frans Olof    male  33.00      0   

     Parch    Ticket      Fare            Cabin Embarked  num_std_cabins  \
27       2     19950  263.0000      C23 C25 C27        S               3   
88       2     19950  263.0000      C23 C25 C27        S               3   
97       1  PC 17759   63.3583          D10 D12        C               2   
118      1  PC 17558  247.5208          B58 B60        C               2   
297      2    113781  151.5500          C22 C26        S               2   
299      1  PC 17558  247.5208          B58 B60        C               2   
305      2    113781  151.5500          C22 C26        S               2   
311      2  PC 17608  262.3750  B57 B59 B63 B66        C               4   
341      2     19950  263.0000      C23 C25 C27        S               3   
390      2    113760  120.0000          B96 B98        S               2   
435      2    113760  120.0000          B96 B98        S               2   
438      4     19950  263.0000      C23 C25 C27        S               3   
498      2    113781  151.5500          C22 C26        S               2   
679      1  PC 17755  512.3292      B51 B53 B55        C               3   
700      0  PC 17757  227.5250          C62 C64        C               2   
742      2  PC 17608  262.3750  B57 B59 B63 B66        C               4   
763      2    113760  120.0000          B96 B98        S               2   
789      0  PC 17593   79.2000          B82 B84        C               2   
802      2    113760  120.0000          B96 B98        S               2   
872      0       695    5.0000      B51 B53 B55        S               3   

     num_let_cabins  
27                0  
88                0  
97                0  
118               0  
297               0  
299               0  
305               0  
311               0  
341               0  
390               0  
435               0  
438               0  
498               0  
679               0  
700               0  
742               0  
763               0  
789               0  
802               0  
872               0  
    \end{verbatim}

    
    Well here's some good news! It looks like non of the people with
duplicate cabins have different letters. So I should be safe to just
separate out the cabins as I planned. I'm going to create a separate
column now, one to store what standard cabin letter they were in and one
to store what letter only cabin they were in.

I'm going to want to make this a function in order to use it again for
testing data, and I'll use the same regex's I had before.

    \begin{Verbatim}[commandchars=\\\{\}]
{\color{incolor}In [{\color{incolor}16}]:} \PY{k}{def} \PY{n+nf}{get\PYZus{}std\PYZus{}cabin\PYZus{}letter}\PY{p}{(}\PY{n}{row}\PY{p}{)}\PY{p}{:}
             \PY{n}{match} \PY{o}{=} \PY{n}{std\PYZus{}format\PYZus{}regex}\PY{o}{.}\PY{n}{match}\PY{p}{(}\PY{n+nb}{str}\PY{p}{(}\PY{n}{row}\PY{p}{[}\PY{l+s+s2}{\PYZdq{}}\PY{l+s+s2}{Cabin}\PY{l+s+s2}{\PYZdq{}}\PY{p}{]}\PY{p}{)}\PY{p}{)}
             \PY{k}{if} \PY{n}{match}\PY{p}{:}
                 \PY{k}{return} \PY{n}{match}\PY{o}{.}\PY{n}{group}\PY{p}{(}\PY{l+m+mi}{1}\PY{p}{)}
             \PY{k}{else}\PY{p}{:}
                 \PY{k}{return} \PY{l+s+s2}{\PYZdq{}}\PY{l+s+s2}{NA}\PY{l+s+s2}{\PYZdq{}}
             
         \PY{k}{def} \PY{n+nf}{get\PYZus{}no\PYZus{}num\PYZus{}cabin\PYZus{}letter}\PY{p}{(}\PY{n}{row}\PY{p}{)}\PY{p}{:}
             \PY{n}{match} \PY{o}{=} \PY{n}{no\PYZus{}num\PYZus{}regex}\PY{o}{.}\PY{n}{match}\PY{p}{(}\PY{n+nb}{str}\PY{p}{(}\PY{n}{row}\PY{p}{[}\PY{l+s+s2}{\PYZdq{}}\PY{l+s+s2}{Cabin}\PY{l+s+s2}{\PYZdq{}}\PY{p}{]}\PY{p}{)}\PY{p}{)}
             \PY{k}{if} \PY{n}{match}\PY{p}{:}
                 \PY{k}{return} \PY{n}{match}\PY{o}{.}\PY{n}{group}\PY{p}{(}\PY{l+m+mi}{1}\PY{p}{)}
             \PY{k}{else}\PY{p}{:}
                 \PY{k}{return} \PY{l+s+s2}{\PYZdq{}}\PY{l+s+s2}{NA}\PY{l+s+s2}{\PYZdq{}}
             
         \PY{n}{df}\PY{p}{[}\PY{l+s+s2}{\PYZdq{}}\PY{l+s+s2}{std\PYZus{}cabin\PYZus{}letter}\PY{l+s+s2}{\PYZdq{}}\PY{p}{]} \PY{o}{=} \PY{n}{df}\PY{o}{.}\PY{n}{apply}\PY{p}{(}\PY{k}{lambda} \PY{n}{row}\PY{p}{:} \PY{n}{get\PYZus{}std\PYZus{}cabin\PYZus{}letter}\PY{p}{(}\PY{n}{row}\PY{p}{)}\PY{p}{,} \PY{n}{axis}\PY{o}{=}\PY{l+m+mi}{1}\PY{p}{)}
         \PY{n}{df}\PY{p}{[}\PY{l+s+s2}{\PYZdq{}}\PY{l+s+s2}{no\PYZus{}num\PYZus{}letter}\PY{l+s+s2}{\PYZdq{}}\PY{p}{]} \PY{o}{=} \PY{n}{df}\PY{o}{.}\PY{n}{apply}\PY{p}{(}\PY{k}{lambda} \PY{n}{row}\PY{p}{:} \PY{n}{get\PYZus{}no\PYZus{}num\PYZus{}cabin\PYZus{}letter}\PY{p}{(}\PY{n}{row}\PY{p}{)}\PY{p}{,} \PY{n}{axis}\PY{o}{=}\PY{l+m+mi}{1}\PY{p}{)}
         \PY{n}{display}\PY{p}{(}\PY{n}{df}\PY{o}{.}\PY{n}{head}\PY{p}{(}\PY{p}{)}\PY{p}{)}
\end{Verbatim}


    
    \begin{verbatim}
   PassengerId  Survived  Pclass  \
0            1     False       3   
1            2      True       1   
2            3      True       3   
3            4      True       1   
4            5     False       3   

                                                Name     Sex   Age  SibSp  \
0                            Braund, Mr. Owen Harris    male  22.0      1   
1  Cumings, Mrs. John Bradley (Florence Briggs Th...  female  38.0      1   
2                             Heikkinen, Miss. Laina  female  26.0      0   
3       Futrelle, Mrs. Jacques Heath (Lily May Peel)  female  35.0      1   
4                           Allen, Mr. William Henry    male  35.0      0   

   Parch            Ticket     Fare Cabin Embarked  num_std_cabins  \
0      0         A/5 21171   7.2500   NaN        S              -1   
1      0          PC 17599  71.2833   C85        C               1   
2      0  STON/O2. 3101282   7.9250   NaN        S              -1   
3      0            113803  53.1000  C123        S               1   
4      0            373450   8.0500   NaN        S              -1   

   num_let_cabins std_cabin_letter no_num_letter  
0              -1               NA            NA  
1               0                C            NA  
2              -1               NA            NA  
3               0                C            NA  
4              -1               NA            NA  
    \end{verbatim}

    
    Alright, I think I'm done with preprocessing mostly. The only thing I
want to do is to one-hot encode the category values, to make random
forests do their things better. I'm also going to normalize the fare
prices using sklearn's \texttt{MinMaxScaler}. I'm not going to do that
with the ages though because their values are in a relatively sane area
already and they contain NaN, which doesn't work with sklearn's
preprocessor. I'm going to one-hot encode \textbf{Pclass} as well, but I
don't know if that will be useful or not.

\textbf{\emph{fuck, you can't have NaN in it.}}

fine i'll just fill it with zero and then normalize.

Finally I want to split the \textbf{Survived} category into a separate
date frame since it is going to be our desired label.

    \begin{Verbatim}[commandchars=\\\{\}]
{\color{incolor}In [{\color{incolor}17}]:} \PY{k+kn}{from} \PY{n+nn}{sklearn}\PY{n+nn}{.}\PY{n+nn}{preprocessing} \PY{k}{import} \PY{n}{MinMaxScaler}
         
         \PY{n}{df\PYZus{}final} \PY{o}{=} \PY{n}{df}
         \PY{c+c1}{\PYZsh{} Initialize a scaler, then apply it to the features}
         \PY{n}{scaler} \PY{o}{=} \PY{n}{MinMaxScaler}\PY{p}{(}\PY{p}{)} \PY{c+c1}{\PYZsh{} default=(0, 1)}
         \PY{n}{df\PYZus{}final}\PY{p}{[}\PY{l+s+s1}{\PYZsq{}}\PY{l+s+s1}{Age}\PY{l+s+s1}{\PYZsq{}}\PY{p}{]} \PY{o}{=} \PY{n}{df}\PY{p}{[}\PY{l+s+s1}{\PYZsq{}}\PY{l+s+s1}{Age}\PY{l+s+s1}{\PYZsq{}}\PY{p}{]}\PY{o}{.}\PY{n}{fillna}\PY{p}{(}\PY{l+m+mi}{0}\PY{p}{)}
         \PY{n}{numerical} \PY{o}{=} \PY{p}{[}\PY{l+s+s1}{\PYZsq{}}\PY{l+s+s1}{Age}\PY{l+s+s1}{\PYZsq{}}\PY{p}{,} \PY{l+s+s1}{\PYZsq{}}\PY{l+s+s1}{Fare}\PY{l+s+s1}{\PYZsq{}}\PY{p}{]}
         
         \PY{n}{df\PYZus{}final}\PY{p}{[}\PY{n}{numerical}\PY{p}{]} \PY{o}{=} \PY{n}{scaler}\PY{o}{.}\PY{n}{fit\PYZus{}transform}\PY{p}{(}\PY{n}{df}\PY{p}{[}\PY{n}{numerical}\PY{p}{]}\PY{p}{)}
         
         \PY{n}{df\PYZus{}final} \PY{o}{=} \PY{n}{pd}\PY{o}{.}\PY{n}{get\PYZus{}dummies}\PY{p}{(}\PY{n}{df}\PY{p}{,} \PY{n}{columns}\PY{o}{=}\PY{p}{[}\PY{l+s+s1}{\PYZsq{}}\PY{l+s+s1}{Pclass}\PY{l+s+s1}{\PYZsq{}}\PY{p}{,} \PY{l+s+s1}{\PYZsq{}}\PY{l+s+s1}{Sex}\PY{l+s+s1}{\PYZsq{}}\PY{p}{,} \PY{l+s+s1}{\PYZsq{}}\PY{l+s+s1}{Embarked}\PY{l+s+s1}{\PYZsq{}}\PY{p}{,} \PY{l+s+s1}{\PYZsq{}}\PY{l+s+s1}{std\PYZus{}cabin\PYZus{}letter}\PY{l+s+s1}{\PYZsq{}}\PY{p}{,} \PY{l+s+s1}{\PYZsq{}}\PY{l+s+s1}{no\PYZus{}num\PYZus{}letter}\PY{l+s+s1}{\PYZsq{}}\PY{p}{]}\PY{p}{)}
         
         \PY{n}{outcome} \PY{o}{=} \PY{n}{df\PYZus{}final}\PY{p}{[}\PY{l+s+s1}{\PYZsq{}}\PY{l+s+s1}{Survived}\PY{l+s+s1}{\PYZsq{}}\PY{p}{]}
         \PY{n}{df\PYZus{}final} \PY{o}{=} \PY{n}{df\PYZus{}final}\PY{o}{.}\PY{n}{drop}\PY{p}{(}\PY{l+s+s1}{\PYZsq{}}\PY{l+s+s1}{Survived}\PY{l+s+s1}{\PYZsq{}}\PY{p}{,} \PY{n}{axis}\PY{o}{=}\PY{l+m+mi}{1}\PY{p}{)}
         
         
         \PY{n}{display}\PY{p}{(}\PY{n}{df\PYZus{}final}\PY{o}{.}\PY{n}{head}\PY{p}{(}\PY{p}{)}\PY{p}{)}
         \PY{n+nb}{print}\PY{p}{(}\PY{n}{df\PYZus{}final}\PY{o}{.}\PY{n}{columns}\PY{p}{)}
\end{Verbatim}


    
    \begin{verbatim}
   PassengerId                                               Name     Age  \
0            1                            Braund, Mr. Owen Harris  0.2750   
1            2  Cumings, Mrs. John Bradley (Florence Briggs Th...  0.4750   
2            3                             Heikkinen, Miss. Laina  0.3250   
3            4       Futrelle, Mrs. Jacques Heath (Lily May Peel)  0.4375   
4            5                           Allen, Mr. William Henry  0.4375   

   SibSp  Parch            Ticket      Fare Cabin  num_std_cabins  \
0      1      0         A/5 21171  0.014151   NaN              -1   
1      1      0          PC 17599  0.139136   C85               1   
2      0      0  STON/O2. 3101282  0.015469   NaN              -1   
3      1      0            113803  0.103644  C123               1   
4      0      0            373450  0.015713   NaN              -1   

   num_let_cabins       ...         std_cabin_letter_C  std_cabin_letter_D  \
0              -1       ...                          0                   0   
1               0       ...                          1                   0   
2              -1       ...                          0                   0   
3               0       ...                          1                   0   
4              -1       ...                          0                   0   

   std_cabin_letter_E  std_cabin_letter_F  std_cabin_letter_G  \
0                   0                   0                   0   
1                   0                   0                   0   
2                   0                   0                   0   
3                   0                   0                   0   
4                   0                   0                   0   

   std_cabin_letter_NA  no_num_letter_D  no_num_letter_F  no_num_letter_NA  \
0                    1                0                0                 1   
1                    0                0                0                 1   
2                    1                0                0                 1   
3                    0                0                0                 1   
4                    1                0                0                 1   

   no_num_letter_T  
0                0  
1                0  
2                0  
3                0  
4                0  

[5 rows x 31 columns]
    \end{verbatim}

    
    \begin{Verbatim}[commandchars=\\\{\}]
Index(['PassengerId', 'Name', 'Age', 'SibSp', 'Parch', 'Ticket', 'Fare',
       'Cabin', 'num\_std\_cabins', 'num\_let\_cabins', 'Pclass\_1', 'Pclass\_2',
       'Pclass\_3', 'Sex\_female', 'Sex\_male', 'Embarked\_C', 'Embarked\_Q',
       'Embarked\_S', 'Embarked\_U', 'std\_cabin\_letter\_A', 'std\_cabin\_letter\_B',
       'std\_cabin\_letter\_C', 'std\_cabin\_letter\_D', 'std\_cabin\_letter\_E',
       'std\_cabin\_letter\_F', 'std\_cabin\_letter\_G', 'std\_cabin\_letter\_NA',
       'no\_num\_letter\_D', 'no\_num\_letter\_F', 'no\_num\_letter\_NA',
       'no\_num\_letter\_T'],
      dtype='object')

    \end{Verbatim}

    \subsection{Parameters}\label{parameters}

Alright, time for the next step of the project! I'm going to throw out
the columns \textbf{PassengerId}, \textbf{Name} and \textbf{Ticket}
right away because there's not really any information to be gained from
them since they are just artificial values, or (in the case of Ticket)
something I can't really make any sense of.

I'm also going to throw out the Cabin column since I've already gotten
everything I needed out of that.

    \begin{Verbatim}[commandchars=\\\{\}]
{\color{incolor}In [{\color{incolor}18}]:} \PY{n}{to\PYZus{}drop} \PY{o}{=} \PY{p}{[}\PY{l+s+s1}{\PYZsq{}}\PY{l+s+s1}{PassengerId}\PY{l+s+s1}{\PYZsq{}}\PY{p}{,} \PY{l+s+s1}{\PYZsq{}}\PY{l+s+s1}{Name}\PY{l+s+s1}{\PYZsq{}}\PY{p}{,} \PY{l+s+s1}{\PYZsq{}}\PY{l+s+s1}{Ticket}\PY{l+s+s1}{\PYZsq{}}\PY{p}{,} \PY{l+s+s1}{\PYZsq{}}\PY{l+s+s1}{Cabin}\PY{l+s+s1}{\PYZsq{}}\PY{p}{]}
         \PY{n}{df\PYZus{}final} \PY{o}{=} \PY{n}{df\PYZus{}final}\PY{o}{.}\PY{n}{drop}\PY{p}{(}\PY{n}{to\PYZus{}drop}\PY{p}{,} \PY{n}{axis}\PY{o}{=}\PY{l+m+mi}{1}\PY{p}{)}
         
         \PY{n}{display}\PY{p}{(}\PY{n}{df\PYZus{}final}\PY{o}{.}\PY{n}{head}\PY{p}{(}\PY{p}{)}\PY{p}{)}
         \PY{n+nb}{print}\PY{p}{(}\PY{n}{df\PYZus{}final}\PY{o}{.}\PY{n}{columns}\PY{p}{)}
\end{Verbatim}


    
    \begin{verbatim}
      Age  SibSp  Parch      Fare  num_std_cabins  num_let_cabins  Pclass_1  \
0  0.2750      1      0  0.014151              -1              -1         0   
1  0.4750      1      0  0.139136               1               0         1   
2  0.3250      0      0  0.015469              -1              -1         0   
3  0.4375      1      0  0.103644               1               0         1   
4  0.4375      0      0  0.015713              -1              -1         0   

   Pclass_2  Pclass_3  Sex_female       ...         std_cabin_letter_C  \
0         0         1           0       ...                          0   
1         0         0           1       ...                          1   
2         0         1           1       ...                          0   
3         0         0           1       ...                          1   
4         0         1           0       ...                          0   

   std_cabin_letter_D  std_cabin_letter_E  std_cabin_letter_F  \
0                   0                   0                   0   
1                   0                   0                   0   
2                   0                   0                   0   
3                   0                   0                   0   
4                   0                   0                   0   

   std_cabin_letter_G  std_cabin_letter_NA  no_num_letter_D  no_num_letter_F  \
0                   0                    1                0                0   
1                   0                    0                0                0   
2                   0                    1                0                0   
3                   0                    0                0                0   
4                   0                    1                0                0   

   no_num_letter_NA  no_num_letter_T  
0                 1                0  
1                 1                0  
2                 1                0  
3                 1                0  
4                 1                0  

[5 rows x 27 columns]
    \end{verbatim}

    
    \begin{Verbatim}[commandchars=\\\{\}]
Index(['Age', 'SibSp', 'Parch', 'Fare', 'num\_std\_cabins', 'num\_let\_cabins',
       'Pclass\_1', 'Pclass\_2', 'Pclass\_3', 'Sex\_female', 'Sex\_male',
       'Embarked\_C', 'Embarked\_Q', 'Embarked\_S', 'Embarked\_U',
       'std\_cabin\_letter\_A', 'std\_cabin\_letter\_B', 'std\_cabin\_letter\_C',
       'std\_cabin\_letter\_D', 'std\_cabin\_letter\_E', 'std\_cabin\_letter\_F',
       'std\_cabin\_letter\_G', 'std\_cabin\_letter\_NA', 'no\_num\_letter\_D',
       'no\_num\_letter\_F', 'no\_num\_letter\_NA', 'no\_num\_letter\_T'],
      dtype='object')

    \end{Verbatim}

    \subsubsection{Cross Validation}\label{cross-validation}

Alright, time to set up cross validation. I'm going to use ten-fold
cross validation because there isn't much data to work with. I'm going
to have sklearn shuffle it first though, because I don't know who
entered the CSV originally so there may be some meaning to it. I am
going to set a random state at the beginning though so I can be
consistent and compare my results as I tweak parameters.

    \begin{Verbatim}[commandchars=\\\{\}]
{\color{incolor}In [{\color{incolor}19}]:} \PY{k+kn}{from} \PY{n+nn}{sklearn}\PY{n+nn}{.}\PY{n+nn}{model\PYZus{}selection} \PY{k}{import} \PY{n}{KFold}
         \PY{k+kn}{from} \PY{n+nn}{collections} \PY{k}{import} \PY{n}{namedtuple}
         
         \PY{n}{K\PYZus{}Split} \PY{o}{=} \PY{n}{namedtuple}\PY{p}{(}\PY{l+s+s1}{\PYZsq{}}\PY{l+s+s1}{K\PYZus{}Split}\PY{l+s+s1}{\PYZsq{}}\PY{p}{,} \PY{p}{[}\PY{l+s+s1}{\PYZsq{}}\PY{l+s+s1}{X\PYZus{}test}\PY{l+s+s1}{\PYZsq{}}\PY{p}{,} \PY{l+s+s1}{\PYZsq{}}\PY{l+s+s1}{X\PYZus{}train}\PY{l+s+s1}{\PYZsq{}}\PY{p}{,} \PY{l+s+s1}{\PYZsq{}}\PY{l+s+s1}{y\PYZus{}test}\PY{l+s+s1}{\PYZsq{}}\PY{p}{,} \PY{l+s+s1}{\PYZsq{}}\PY{l+s+s1}{y\PYZus{}train}\PY{l+s+s1}{\PYZsq{}}\PY{p}{]}\PY{p}{)}
         
         \PY{n}{RANDOM\PYZus{}STATE} \PY{o}{=} \PY{l+m+mi}{1337}
         
         \PY{n}{kf} \PY{o}{=} \PY{n}{KFold}\PY{p}{(}\PY{n}{n\PYZus{}splits}\PY{o}{=}\PY{l+m+mi}{10}\PY{p}{,} \PY{n}{shuffle}\PY{o}{=}\PY{k+kc}{True}\PY{p}{,} \PY{n}{random\PYZus{}state}\PY{o}{=}\PY{n}{RANDOM\PYZus{}STATE}\PY{p}{)}
         
         \PY{n}{X} \PY{o}{=} \PY{n}{df\PYZus{}final}
         \PY{n}{y} \PY{o}{=} \PY{n}{outcome}
\end{Verbatim}


    \subsubsection{Accuracy}\label{accuracy}

So in order to calculate the accuracy, I'm going to use the
\texttt{accuracy\_score} from \texttt{sklearn} because that's what I've
used before when working with data like this. Since it doesn't really
matter whether I correctly predict someone living or dying, the accuracy
should be fine.

In order to have a good baseline, I'm going to use the actual percentage
of who survived the Titanic, which is easy enough. However, I can't use
everything being categorized as correct for this, because then the
accuracy and fbeta will just be 1! So instead I'm going to make the
naive model kill everyone, since most people on the titanic died.

The Positive will be someone surviving and the negative will be someone
not surviving in the real life scenario, but since I can't do that with
precision scores,

Then I'm going to have to calculate the first accuracy and fbeta scores
myself since sklearn can't do that.

    \begin{Verbatim}[commandchars=\\\{\}]
{\color{incolor}In [{\color{incolor}21}]:} \PY{n}{naive\PYZus{}TP} \PY{o}{=} \PY{l+m+mi}{0}
         \PY{n}{naive\PYZus{}FP} \PY{o}{=} \PY{l+m+mi}{0}
         
         \PY{n}{naive\PYZus{}TN} \PY{o}{=} \PY{n}{outcome}\PY{o}{.}\PY{n}{count}\PY{p}{(}\PY{p}{)} \PY{o}{\PYZhy{}} \PY{n}{np}\PY{o}{.}\PY{n}{sum}\PY{p}{(}\PY{n}{outcome}\PY{p}{)}
         \PY{n}{naive\PYZus{}FN} \PY{o}{=} \PY{n}{np}\PY{o}{.}\PY{n}{sum}\PY{p}{(}\PY{n}{outcome}\PY{p}{)}
         
         \PY{n}{naive\PYZus{}accuracy} \PY{o}{=} \PY{p}{(}\PY{n}{naive\PYZus{}TP} \PY{o}{+} \PY{n}{naive\PYZus{}TN}\PY{p}{)} \PY{o}{/} \PY{p}{(}\PY{n}{naive\PYZus{}TP} \PY{o}{+} \PY{n}{naive\PYZus{}FP} \PY{o}{+} \PY{n}{naive\PYZus{}TN} \PY{o}{+} \PY{n}{naive\PYZus{}FN}\PY{p}{)}
         \PY{n}{lazy\PYZus{}max\PYZus{}accuracy} \PY{o}{=} \PY{o}{.}\PY{l+m+mi}{70}
         
         \PY{c+c1}{\PYZsh{} Print the results }
         \PY{n+nb}{print}\PY{p}{(}\PY{l+s+s2}{\PYZdq{}}\PY{l+s+s2}{Naive Predictor: [Accuracy score: }\PY{l+s+si}{\PYZob{}:.4f\PYZcb{}}\PY{l+s+s2}{]}\PY{l+s+s2}{\PYZdq{}}\PY{o}{.}\PY{n}{format}\PY{p}{(}\PY{n}{naive\PYZus{}accuracy}\PY{p}{)}\PY{p}{)}
\end{Verbatim}


    \begin{Verbatim}[commandchars=\\\{\}]
Naive Predictor: [Accuracy score: 0.6162]

    \end{Verbatim}

    That's some neat accuracy, espectally since the accuracy you put in the
project description was just 70\%.

So time to do some stuff. I want to make a generic function that I can
use to train any model and any split. So lets make that.

    \begin{Verbatim}[commandchars=\\\{\}]
{\color{incolor}In [{\color{incolor}22}]:} \PY{k+kn}{from} \PY{n+nn}{sklearn}\PY{n+nn}{.}\PY{n+nn}{metrics} \PY{k}{import} \PY{n}{accuracy\PYZus{}score}
         
         \PY{c+c1}{\PYZsh{} Returns the accuracy score.}
         \PY{k}{def} \PY{n+nf}{train\PYZus{}test\PYZus{}single\PYZus{}split}\PY{p}{(}\PY{n}{learner}\PY{p}{,} \PY{n}{X\PYZus{}train}\PY{p}{,} \PY{n}{X\PYZus{}test}\PY{p}{,} \PY{n}{y\PYZus{}train}\PY{p}{,} \PY{n}{y\PYZus{}test}\PY{p}{)}\PY{p}{:}
             \PY{n}{learner} \PY{o}{=} \PY{n}{learner}\PY{o}{.}\PY{n}{fit}\PY{p}{(}\PY{n}{X\PYZus{}train}\PY{p}{,} \PY{n}{y\PYZus{}train}\PY{p}{)}
             \PY{n}{predictions} \PY{o}{=} \PY{n}{learner}\PY{o}{.}\PY{n}{predict}\PY{p}{(}\PY{n}{X\PYZus{}test}\PY{p}{)}
             \PY{k}{return} \PY{n}{accuracy\PYZus{}score}\PY{p}{(}\PY{n}{y\PYZus{}test}\PY{p}{,} \PY{n}{predictions}\PY{p}{)}
         
         \PY{k}{def} \PY{n+nf}{train\PYZus{}test\PYZus{}kfold}\PY{p}{(}\PY{n}{learner}\PY{p}{,} \PY{n}{X}\PY{p}{,} \PY{n}{y}\PY{p}{,} \PY{n}{kf}\PY{p}{)}\PY{p}{:}
             \PY{n}{scores} \PY{o}{=} \PY{p}{[}\PY{p}{]}
             \PY{k}{for} \PY{n}{train\PYZus{}index}\PY{p}{,} \PY{n}{test\PYZus{}index} \PY{o+ow}{in} \PY{n}{kf}\PY{o}{.}\PY{n}{split}\PY{p}{(}\PY{n}{X}\PY{p}{)}\PY{p}{:}
                 \PY{n}{X\PYZus{}train}\PY{p}{,} \PY{n}{y\PYZus{}train} \PY{o}{=} \PY{n}{X}\PY{o}{.}\PY{n}{iloc}\PY{p}{[}\PY{n}{train\PYZus{}index}\PY{p}{]}\PY{p}{,} \PY{n}{y}\PY{o}{.}\PY{n}{iloc}\PY{p}{[}\PY{n}{train\PYZus{}index}\PY{p}{]}
                 \PY{n}{X\PYZus{}test}\PY{p}{,} \PY{n}{y\PYZus{}test} \PY{o}{=} \PY{n}{X}\PY{o}{.}\PY{n}{iloc}\PY{p}{[}\PY{n}{test\PYZus{}index}\PY{p}{]}\PY{p}{,} \PY{n}{y}\PY{o}{.}\PY{n}{iloc}\PY{p}{[}\PY{n}{test\PYZus{}index}\PY{p}{]}
                 \PY{n}{single\PYZus{}score} \PY{o}{=} \PY{n}{train\PYZus{}test\PYZus{}single\PYZus{}split}\PY{p}{(}\PY{n}{learner}\PY{p}{,} \PY{n}{X\PYZus{}train}\PY{p}{,} \PY{n}{X\PYZus{}test}\PY{p}{,} \PY{n}{y\PYZus{}train}\PY{p}{,} \PY{n}{y\PYZus{}test}\PY{p}{)}
                 \PY{n}{scores} \PY{o}{+}\PY{o}{=} \PY{p}{[}\PY{n}{single\PYZus{}score}\PY{p}{]}
             \PY{k}{return} \PY{n+nb}{sum}\PY{p}{(}\PY{n}{scores}\PY{p}{)} \PY{o}{/} \PY{n+nb}{len}\PY{p}{(}\PY{n}{scores}\PY{p}{)}
\end{Verbatim}


    Okay lets actually do the stuff now. I'm running out of time :(

I'm going to use some of sklearn's fancy libraries to do the work for
me.

\href{https://i.imgur.com/jwBKz6t.jpg}{This} is basically me. Minus the
google part, rip host matching. oh and minus the not sklearn thing, that
too. but same deal.

Okay I hope my laptop doesn't catch on fire here.

    \begin{Verbatim}[commandchars=\\\{\}]
{\color{incolor}In [{\color{incolor}27}]:} \PY{k+kn}{from} \PY{n+nn}{sklearn}\PY{n+nn}{.}\PY{n+nn}{ensemble} \PY{k}{import} \PY{n}{RandomForestClassifier}
         
         \PY{k+kn}{from} \PY{n+nn}{sklearn}\PY{n+nn}{.}\PY{n+nn}{model\PYZus{}selection} \PY{k}{import} \PY{n}{GridSearchCV}
         \PY{k+kn}{from} \PY{n+nn}{sklearn}\PY{n+nn}{.}\PY{n+nn}{metrics} \PY{k}{import} \PY{n}{make\PYZus{}scorer}
\end{Verbatim}


    \begin{Verbatim}[commandchars=\\\{\}]
{\color{incolor}In [{\color{incolor}28}]:} \PY{k+kn}{from} \PY{n+nn}{sklearn}\PY{n+nn}{.}\PY{n+nn}{ensemble} \PY{k}{import} \PY{n}{RandomForestClassifier}
         
         \PY{k+kn}{from} \PY{n+nn}{sklearn}\PY{n+nn}{.}\PY{n+nn}{model\PYZus{}selection} \PY{k}{import} \PY{n}{GridSearchCV}
         \PY{k+kn}{from} \PY{n+nn}{sklearn}\PY{n+nn}{.}\PY{n+nn}{metrics} \PY{k}{import} \PY{n}{make\PYZus{}scorer}
         
         \PY{n}{clf} \PY{o}{=} \PY{n}{RandomForestClassifier}\PY{p}{(}\PY{n}{random\PYZus{}state}\PY{o}{=}\PY{n}{RANDOM\PYZus{}STATE}\PY{p}{)}
         
         \PY{c+c1}{\PYZsh{} tbh i\PYZsq{}d rather use fbeta but w/e}
         \PY{n}{scorer} \PY{o}{=} \PY{n}{make\PYZus{}scorer}\PY{p}{(}\PY{n}{accuracy\PYZus{}score}\PY{p}{)}
         
         \PY{n}{parameters} \PY{o}{=} \PY{p}{\PYZob{}}
             \PY{l+s+s1}{\PYZsq{}}\PY{l+s+s1}{n\PYZus{}estimators}\PY{l+s+s1}{\PYZsq{}}\PY{p}{:} \PY{p}{[}\PY{l+m+mi}{5}\PY{p}{,} \PY{l+m+mi}{10}\PY{p}{,} \PY{l+m+mi}{15}\PY{p}{,} \PY{l+m+mi}{25}\PY{p}{]}\PY{p}{,}
             \PY{l+s+s1}{\PYZsq{}}\PY{l+s+s1}{criterion}\PY{l+s+s1}{\PYZsq{}}\PY{p}{:} \PY{p}{[}\PY{l+s+s1}{\PYZsq{}}\PY{l+s+s1}{gini}\PY{l+s+s1}{\PYZsq{}}\PY{p}{,} \PY{l+s+s1}{\PYZsq{}}\PY{l+s+s1}{entropy}\PY{l+s+s1}{\PYZsq{}}\PY{p}{]}\PY{p}{,}
             \PY{l+s+s1}{\PYZsq{}}\PY{l+s+s1}{max\PYZus{}features}\PY{l+s+s1}{\PYZsq{}}\PY{p}{:} \PY{p}{[}\PY{l+s+s1}{\PYZsq{}}\PY{l+s+s1}{auto}\PY{l+s+s1}{\PYZsq{}}\PY{p}{,} \PY{l+s+s1}{\PYZsq{}}\PY{l+s+s1}{sqrt}\PY{l+s+s1}{\PYZsq{}}\PY{p}{,} \PY{l+s+s1}{\PYZsq{}}\PY{l+s+s1}{log2}\PY{l+s+s1}{\PYZsq{}}\PY{p}{,} \PY{k+kc}{None}\PY{p}{]}\PY{p}{,}
             \PY{l+s+s1}{\PYZsq{}}\PY{l+s+s1}{max\PYZus{}depth}\PY{l+s+s1}{\PYZsq{}}\PY{p}{:} \PY{p}{[}\PY{k+kc}{None}\PY{p}{,} \PY{l+m+mi}{2}\PY{p}{,} \PY{l+m+mi}{4}\PY{p}{,} \PY{l+m+mi}{8}\PY{p}{]}
         \PY{p}{\PYZcb{}}
         
         \PY{n+nb}{print}\PY{p}{(}\PY{l+s+s2}{\PYZdq{}}\PY{l+s+s2}{did the easy shit}\PY{l+s+s2}{\PYZdq{}}\PY{p}{)}
         
         
         
         \PY{c+c1}{\PYZsh{} yeah idk if anything else should be added, knock yourself out.}
         
         \PY{c+c1}{\PYZsh{} tfw you forget the random state the first time}
         \PY{c+c1}{\PYZsh{} tfw gridsearch doesn\PYZsq{}t actually take a random state}
         \PY{n}{grid} \PY{o}{=} \PY{n}{GridSearchCV}\PY{p}{(}\PY{n}{clf}\PY{p}{,} \PY{n}{param\PYZus{}grid}\PY{o}{=}\PY{n}{parameters}\PY{p}{,} \PY{n}{scoring}\PY{o}{=}\PY{n}{scorer}\PY{p}{,} \PY{n}{cv}\PY{o}{=}\PY{n}{kf}\PY{p}{)}
         
         \PY{n+nb}{print}\PY{p}{(}\PY{l+s+s2}{\PYZdq{}}\PY{l+s+s2}{made the intense thing}\PY{l+s+s2}{\PYZdq{}}\PY{p}{)}
         
         \PY{n}{grid}\PY{o}{.}\PY{n}{fit}\PY{p}{(}\PY{n}{X}\PY{p}{,} \PY{n}{y}\PY{p}{)}
         
         \PY{n+nb}{print}\PY{p}{(}\PY{l+s+s2}{\PYZdq{}}\PY{l+s+s2}{finished the intense thing}\PY{l+s+s2}{\PYZdq{}}\PY{p}{)}
         
         \PY{c+c1}{\PYZsh{} nearly named this the \PYZdq{}chad\PYZus{}clf\PYZdq{}}
         \PY{n}{best\PYZus{}clf} \PY{o}{=} \PY{n}{grid}\PY{o}{.}\PY{n}{best\PYZus{}estimator\PYZus{}}
         
         \PY{c+c1}{\PYZsh{} Don\PYZsq{}t need to do this beacuse the gridsearch keeps track of the best estimator results.}
         \PY{c+c1}{\PYZsh{}best\PYZus{}accuracy = train\PYZus{}test\PYZus{}kfold(best\PYZus{}clf, X, y, kf)}
         
         \PY{c+c1}{\PYZsh{}print(\PYZdq{}Best accuracy:\PYZdq{}, best\PYZus{}accuracy)}
\end{Verbatim}


    \begin{Verbatim}[commandchars=\\\{\}]
did the easy shit
made the intense thing
finished the intense thing

    \end{Verbatim}

    yeah okay that's good enough for tonight. confusion matricies are
overrated, idk even know how to do that for kfold.

oh yeah my computer is not on fire but it is certainly not happy, i've
been waiting like 10 minutes now or something idk.

yeah i can get the best parameters too if this \textbf{ever finishes
running}

holy fuck it just finished with a FUCKING ERROR

okay so apparently grid search doesn't take a random\_state that is...
annoying.

any way, i reduced the number of possible parameters so hopefully it
will actually finish this time.

    \begin{Verbatim}[commandchars=\\\{\}]
{\color{incolor}In [{\color{incolor}29}]:} \PY{n}{best\PYZus{}params} \PY{o}{=} \PY{n}{grid}\PY{o}{.}\PY{n}{best\PYZus{}params\PYZus{}}
         \PY{n}{best\PYZus{}score} \PY{o}{=} \PY{n}{grid}\PY{o}{.}\PY{n}{best\PYZus{}score\PYZus{}}
         
         \PY{n+nb}{print}\PY{p}{(}\PY{l+s+s2}{\PYZdq{}}\PY{l+s+s2}{The GridSearchCV found the best number of estimators to be: }\PY{l+s+si}{\PYZob{}\PYZcb{}}\PY{l+s+s2}{, the best criterion to be: }\PY{l+s+si}{\PYZob{}\PYZcb{}}\PY{l+s+s2}{, the best max\PYZus{}features to be: }\PY{l+s+si}{\PYZob{}\PYZcb{}}\PY{l+s+s2}{ and the best max\PYZus{}depth to be: }\PY{l+s+si}{\PYZob{}\PYZcb{}}\PY{l+s+s2}{\PYZdq{}}\PY{o}{.}\PY{n}{format}\PY{p}{(}\PY{n}{best\PYZus{}params}\PY{p}{[}\PY{l+s+s1}{\PYZsq{}}\PY{l+s+s1}{n\PYZus{}estimators}\PY{l+s+s1}{\PYZsq{}}\PY{p}{]}\PY{p}{,} \PY{n}{best\PYZus{}params}\PY{p}{[}\PY{l+s+s1}{\PYZsq{}}\PY{l+s+s1}{criterion}\PY{l+s+s1}{\PYZsq{}}\PY{p}{]}\PY{p}{,} \PY{n}{best\PYZus{}params}\PY{p}{[}\PY{l+s+s1}{\PYZsq{}}\PY{l+s+s1}{max\PYZus{}features}\PY{l+s+s1}{\PYZsq{}}\PY{p}{]}\PY{p}{,} \PY{n}{best\PYZus{}params}\PY{p}{[}\PY{l+s+s1}{\PYZsq{}}\PY{l+s+s1}{max\PYZus{}depth}\PY{l+s+s1}{\PYZsq{}}\PY{p}{]}\PY{p}{)}\PY{p}{)}
         \PY{n+nb}{print}\PY{p}{(}\PY{l+s+s2}{\PYZdq{}}\PY{l+s+s2}{The final GridSearch RandomForest accuracy was: }\PY{l+s+si}{\PYZob{}:.4f\PYZcb{}}\PY{l+s+s2}{\PYZdq{}}\PY{o}{.}\PY{n}{format}\PY{p}{(}\PY{n}{best\PYZus{}score}\PY{p}{)}\PY{p}{)}
\end{Verbatim}


    \begin{Verbatim}[commandchars=\\\{\}]
The GridSearchCV found the best number of estimators to be: 25, the best criterion to be: entropy, the best max\_features to be: None and the best max\_depth to be: 8
The final GridSearch RandomForest accuracy was: 0.8272

    \end{Verbatim}

    \subsection{Final Result}\label{final-result}

It finally fucking finished. I had reduced a bunch of the possible
parameters so that may have been it. The ones I found with the random
state of 1337 decided that a Random Forest with the following parameters
was best.

\begin{itemize}
\tightlist
\item
  \textbf{n\_estimators}: 25
\item
  wow, how surprising, it's the maximum amount /s
\item
  \textbf{criterion}: entropy
\item
  this was actually kind of surprising since sklearn's default is gini.
  but i guess we learned this in class.
\item
  \textbf{max\_features}: None
\item
  how surprising, it wants more of my cpu /s
\item
  \textbf{max\_depth}: 8
\item
  This is actually surprising since 8 is \textbf{not} the maximum depth
  it could choose, that would be None. But it is the max among the
  numbers I provided.
\end{itemize}

I've tried to find a way to create a confusion matrix or some graphs for
this, but I don't think it's feasible to do while using grid search.
especially since I still want to run the entire notebook kernel fresh.
Plus I think it's going to rain soon so i g2g. enjoy.


    % Add a bibliography block to the postdoc
    
    
    
    \end{document}
