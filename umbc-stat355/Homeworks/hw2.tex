\documentclass[12pt]{article}
\setlength{\oddsidemargin}{0in}
\setlength{\evensidemargin}{0in}
\setlength{\textwidth}{6.5in}
\setlength{\parindent}{0in}
\setlength{\parskip}{\baselineskip}

\usepackage{amsmath,amsfonts,amssymb}

\title{STAT 355 - HW2}

\begin{document}

STAT 355 Spring 2018\hfill Homework \#2\\
Robert Rose (02/13)

\hrulefill

\begin{enumerate}
\item \textbf{2.5.71} An oil exploration company currently has two active projects, one in Asia and the other in Europe. Let $A$ be the even that the Asian project is successful and $B$ be the event that the European project is successful. Suppose that $A$ and $B$ are independent events with $P(A) = .4$ and $P(B) = .7$.
  \begin{enumerate}
  \item If the Asian project is not successful, what is the probability that the European project is also not successful? Explain your reasoning.
  \item What is the probability that at least one of the two projects will be successful?
  \item Given that at least one of the two projects is successful, what is the probability that only the Asian project is successful?
  \end{enumerate}
\newpage

\item \textbf{2.5.87} Consider randomly selecting a single individual and having that person test drive 3 different vehicles. Define events A\textsubscript{1}, A\textsubscript{2}, and A\textsubscript{3} by
\begin{align*}
&A_1 = \text{likes vehicle \#1} & A_2 = \text{likes vehicle \#2}\\
&A_3 = \text{likes vehicle \#3} &
\end{align*}
Suppose that $P(A_1) = .55, P(A_2) = .65, P(A_3) = .70, P(A_1 \cup A_2) = .80, P(A_2 \cap A_3) = .40$, and $P(A_1 \cup A_2 \cup A_3) = .88.$
  \begin{enumerate}
  \item What is the probability that the individual likes both vehicle \#1 vehicle \#2?
  \item Determine and interpret $P(A_2 \mid A_3)$.
  \item Are A\textsubscript{2} and A\textsubscript{3} independent events? Answer in two different ways.
  \item If you learn that the individual did not like vehicle \#1, what now is the probability that he/she liked at least one of the other two vehicles?
  \end{enumerate}

\newpage

\item \textbf{3.1.5} If the sample space $S$ is an infinite set, does this necessarily imply that any rv $X$ defined from $S$ will have an infinite set of possible values? If yes, say why. If no, give an example.


\newpage

\item \textbf{3.2.15} Many manufacturers have quality control programs that include inspection of incoming materials for defects. Suppose a computer manufacturer receives circuit boards in batches of five. Two boards are selected from each batch for inspection. We can represent possible outcomes of the selection process by pairs. For example, the pair $(1, 2)$ represents the selection of boards 1 and 2 for inspection.
  \begin{enumerate}
  \item List the ten different possible outcomes.
  \item Suppose that boards 1 and 2 ar the only defective boards in a batch. Two boards are chosen at random. Define $X$ to be the number of defective boards observed among those inspected. Find the probability distribution of $X$.
  \item Let $F(x)$ denote the cdf of $X$. First determine $F(0) = P(X\leq0)$. $F(1)$, and $F(2)$; then obtain $F(x)$ for all other x.
  \end{enumerate}

\end{enumerate}
\end{document}